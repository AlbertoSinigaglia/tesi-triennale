% !TEX encoding = UTF-8
% !TEX TS-program = pdflatex
% !TEX root = ../tesi.tex

%**************************************************************
\chapter{Conclusioni}
\label{cap:conclusioni}
%**************************************************************
\intro{In questo capitolo conclusivo viene analizzato retrospettivamente il progetto di stage, focalizzandosi sul raggiungimento degli obiettivi, sulle conoscenze acquisite e/o rinforzate, e su una valutazione personale di questo percorso.}
%**************************************************************
\section{Consuntivo finale}
	Nel complesso, lo stage ha avuto una durata di esattamente 320 ore come preventivate da piano di lavoro, con conclusione il 10/06/2021 con una presentazione e demo a vari componenti e tutor aziendale.\\


%**************************************************************
\section{Raggiungimento degli obiettivi}
	Come descritto nei capitoli precedenti, il prodotto soddisfa tutti i requisiti, sia obbligatori che desiderabili, identificati durante la fase di analisi dei requisiti. \\
	Va oltre a tali requisiti con Grafana e Docker, in quanto non erano stati individuati come requisiti all'inizio, ma le tempistiche hanno permesso anche il loro completamento e rilascio.

%**************************************************************
\section{Conoscenze acquisite}
	Per la realizzazione di questo progetto son state fondamentali le nozioni apprese durante il corso di studi, in particolare il corso Ingegneria del Software, per la gestione del progetto (da un punto di vista teorico) e per la parte di codifica (dal punto di vista pratico), sopratutto la conoscenza dei design pattern per rendere modulare e estendibile il progetto, e il corso di Basi di Dati, per la creazione e progettazione della base di dati. \\
	Oltre alle conoscenze apprese grazie al corso di studi, son state fondamentali anche conoscenze extra, apprese durante lo stage, come:
	\begin{itemize}
		\item Spring: essendo un framework usato globalmente, nei modi più disparati, è estremamente grande e relativamente complesso rispetto a suoi concorrenti di altri linguaggi, ma è stato fondamentale per lo stage per lo sviluppo dell'applicativo
		\item Docker: applicativo utile per il deploy, anch'esso però relativamente complesso a priva vista, fondamentale per il rilascio dell'applicativo sviluppato
		\item Grafana: programma altamente customizzabile, facile da usare, con un sacco di feature pronte out-of-the-box, essenziale per la creazione della dashboard di analisi della base di dati
		\item PostgreSQL: per quanto anch'esso sia di base SQL, come MySQL, son state necessarie nozioni extra per il suo uso per la realizzazione della base di dati del progetto 
	\end{itemize}
	

%**************************************************************
\section{Valutazione personale}
	Il progetto di stage offerto da Euronovate, a mio parere, si è svolto in modo ottimo. Per quanto non elementare, è stato correttamente ponderato per le 320 ore lavorative previste, e l'azienda e il tutor, Matteo Gnoato, per quanto in un momento abbastanza impegnativo, son sempre stati disponibili per chiarimenti o incontri in caso di problemi. \\
	Personalmente ho trovato la parte iniziale, fino alla progettazione tecnica,  un po lenta, in quanto ancora improntata su un modello "in presenza", e che quindi richiedesse la configurazione della postazione di lavoro, e avendo un po di dimestichezza già con gli strumenti usati durante lo stage, anche la parte di esplorazione delle tecnologie è stata un po lenta, ma per il resto non c'è stato alcun problema. \\
	Per quanto mi sia sempre piaciuto programmare, durante lo stage ho capito che in se, programmare tutto il giorno, non è veramente la mia più grande passione, e questo sicuramente giocherà un ruolo importante nella scelta del corso magistrale che deciderò di intraprendere.

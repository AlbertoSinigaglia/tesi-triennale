% !TEX encoding = UTF-8
% !TEX TS-program = pdflatex
% !TEX root = ../tesi.tex

%**************************************************************
\chapter{Introduzione}
\label{cap:introduzione}
%**************************************************************

Il seguente capitolo vuole introdurre brevemente il progetto affrontato e l'azienda ospitante ideatrice di esso. \\



\iffalse
\noindent Esempio di utilizzo di un termine nel glossario \\
\gls{api}. \\
\noindent Esempio di citazione in linea \\
\cite{site:agile-manifesto}. \\
\noindent Esempio di citazione nel pie' di pagina \\
citazione\footcite{womak:lean-thinking} \\
\fi


%**************************************************************
\section{L'azienda}

\par EsignWorld è un'azienda italiana facente parte del gruppo Euronovate, gruppo del settore IT che conta circa 120 collaboratori, con sede centrale in Svizzera con varie sedi in Spagna, Italia, Romania e Cina. 
\newline
\newline
\par EsignWorld si occupa di sviluppare e progettare soluzioni complesse nell'ambito della digital identity, digital signature e digital onboarding, sia software che hardware riguardanti i processi di dematerializzazione all'interno delle aziende.
\newline
\newline
\par Durante lo stage ci si è andati ad interfacciare con Redmine, che per quanto non sia un prodotto sviluppato direttamente da Euronovate, è utilizzato internamente per la pianificazione di progetti e per il tracciamento delle segnalazioni di bug.
\newline
\begin{figure}[!h] 
	\centering 
	\includegraphics[width=0.5\columnwidth]{logo-azienda.png} 
	\caption{Logo dell'azienda}
\end{figure}
\begin{figure}[!h] 
	\centering 
	\includegraphics[width=0.5\columnwidth]{logo-redmine.png} 
	\caption{Logo di Redmine}
\end{figure}
\newpage
%**************************************************************
\section{L'idea}

Lo stage consiste nello sviluppo di un prodotto software per il monitoraggio delle segnalazioni, con conseguente notifica in caso di violazioni dei service level agreement stipulati tra l'azienda e i clienti.\\
Per raggiungere tale obiettivo, ci si è appoggiati a Redmine, software adottato dall'azienda per la gestione delle segnalazioni e l'ottenimento dei dati necessari alle analisi. Redmine infatti raccoglie tutti i dati fondamentali, quali segnalazioni, tempo dedicato per ogni task e un log di tutti i cambiamenti avvenuti. \\
Il prodotto periodicamente quindi scarica gli aggiornamenti da Redmine, li analizza, e tramite delle policy predefinite per ogni cliente, va a notificare tramite vari canali l'eventuale violazione di esse. \\
Infine, l'applicativo espone un'API per l'ottenimento di informazioni statistiche riguardanti le violazioni. 

%**************************************************************
\section{Organizzazione del testo}

\begin{description}
    \item[{\hyperref[cap:descrizione-stage]{Il secondo capitolo}}] espone una panoramica sugli obbiettivi dello stage e la sua pianificazione, con un'analisi preventiva dei rischi.
    
    \item[{\hyperref[cap:analisi-requisiti]{Il terzo capitolo}}] descrive l'analisi dei requisiti del progetto affrontato nello stage, durante la quale son state definite le funzionalità desiderate.
    
    \item[{\hyperref[cap:progettazione-codifica]{Il quarto capitolo}}] approfondisce la progettazione e la codifica del progetto, esponendo successivamente le problematiche incontrate
    
    \item[{\hyperref[cap:verifica-validazione]{Il quinto capitolo}}] 
    espone tutte le verifiche effettaute sul progetto durante il suo sviluppo e la validazione finale, per garantire un buon risultato finale, conforme con i requisiti individuati
    
    \item[{\hyperref[cap:documentazione]{Il sesto capitolo}}] descive il processo di documentazione intrapreso durante lo stage
    
    \item[{\hyperref[cap:conclusioni]{Il settimo capitolo}}] espone le conclusioni tratte dallo stage, con annesse conoscenze acquisite e consuntivo finale
\end{description}

\noindent Riguardo la stesura del testo, relativamente al documento sono state adottate le seguenti convenzioni tipografiche:
\begin{itemize}
	\item gli acronimi, le abbreviazioni e i termini ambigui o di uso non comune menzionati vengono definiti nel glossario, situato alla fine del presente documento;
	\item per la prima occorrenza dei termini riportati nel glossario viene utilizzata la seguente nomenclatura: \gloxy{parola};
	\item i termini in lingua straniera o facenti parti del gergo tecnico sono evidenziati con il carattere \emph{corsivo}.
\end{itemize}
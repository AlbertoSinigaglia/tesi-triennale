% !TEX encoding = UTF-8
% !TEX TS-program = pdflatex
% !TEX root = ../tesi.tex

%**************************************************************
\chapter{Descrizione dello stage}
\label{cap:descrizione-stage}
%**************************************************************

\intro{In questa sezione verrà trattato come si è svolto lo stage presso EsignWorld, considerando gli obiettivi pianificati, discutendo dei possibili rischi e la comunicazione con il tutor aziendale.}\\


%**************************************************************
\section{Analisi preventiva dei rischi}

Durante la fase di analisi iniziale sono stati individuati alcuni possibili rischi a cui si potrà andare incontro.
Si è quindi proceduto a elaborare delle possibili soluzioni per far fronte ad essi.\\


\begin{risk}{Tecnologie nuove}
    \riskdescription{le tecnologie consigliate per lo sviluppo del progetto erano nuove, causando un'inevitabile inesperienza nel loro utilizzo}
    \risksolution{è stato predefinito un periodo iniziale di formazione personale per le tecnologie che saranno poi utilizzate per lo sviluppo, così da aver tempo di familiarizzarci leggermente e arrivare alla codifica con le nozioni base già apprese}
\end{risk}
\begin{risk}{Sistemi esterni}
	\riskdescription{il progetto richiede di interfacciarsi con sistemi esterni (quali per esempio Redmine) mai usati prima e che potrebbero avere una documentazione non esaustiva}
	\risksolution{è stata predefinito un periodo iniziale di formazione personale sui sistemi ai quali ci si interfaccerà durante la codifica, così da avere già una chiara visione del loro funzionamento e delle feature che mettono a disposizione}
\end{risk}
\begin{risk}{Gestione dello smart working}
	\riskdescription{lo stage si è tenuto interamente da remoto, portando quindi a un potenziale rischio di mancanza di comunicazione, che avrebbe potuto portare a un'incerta comprensione degli obbiettivi dello stage e del progetto affrontato}
	\risksolution{si è deciso di avere un meeting giornaliero con il tutor aziendale nelle prime fasi del progetto, così da esser sicuri che si arrivasse alle fasi finali (come la codifica) con tutti i chiarimenti necessari espletati}
\end{risk}
\newpage 
%**************************************************************
\section{Requisiti e obiettivi}
\begin{itemize}
	\item Obbligatori:
		\begin{enumerate}
			\item Studio e valutazione di componenti esterni.
			\item Test e recupero dati dai servizi esposti da Redmine.
			\item Analisi di vari sistemi di notifica.
			\item Redazione di un documento di Analisi dei Requisiti.
			\item Redazione di un documento di Progettazione Tecnica.
			\item Sviluppo/Codifica del progetto seguendo le norme di codifica aziendali.
			\item Redazione documentazione del progetto:
				\begin{enumerate}
					\item Manuale di manutenzione.
					\item Manuale di installazione.
					\item Documentazione API.
				\end{enumerate}
		\end{enumerate}
	\item Opzionali (da valutare durante il periodo di Analisi dei Requisiti):
	\begin{enumerate}
		\item Integrazione dell'API con l'applicativo ENAnalytics
		\item Integrazione della base di dati con l'applicativo Grafana
		\item Sviluppo di test di unità e di integrità
	\end{enumerate}
\end{itemize}

%**************************************************************
\section{Pianificazione}
In questa sezione sono descritte le attività pianificate per il progetto di stage, insieme alla ripartizione delle ore per ognuna di esse.

	\subsection{Fasi del progetto}
		Di seguito sono elencate le varie fasi previste per lo stage e una loro stima oraria, in ordine cronologico:
		\begin{enumerate}
			\item \textbf{Conoscenze generali}: approfondimento e installazione degli ambienti di sviluppo e di versionamento, e abilitazione strumenti aziendali. 
			\item \textbf{Acquisizione degli standard aziendali}: esposizione dei servizi server Euronovate e dell'architettura della soluzione di firma Euronovate. 
			\item \textbf{Formazione personale}: formazione sul framework target per lo sviluppo, valutazione e test di componenti esterni (sistemi di notifica come email e messaggistica istantanea), istruzione sull'applicativo Redmine e recupero dati dai suoi servizi esposti.
			\item \textbf{Analisi dei requisiti}: analisi del tipo di informazioni da monitorare e successivamente da esporre/pubblicare, selezione dei sistemi e delle regole di notifica (e relativa frequenza di aggiornamento dati locali).
			\item \textbf{Progettazione tecnica}: stesura della specifica tecnica del Back-End, dei servizi esposti e degli oggetti di scambio.
			\item \textbf{Codifica}: sviluppo del prodotto seguendo le norme di codifica aziendali e potenzialmente test di unità e integrità.
			\item \textbf{Documentazione}: stesura manuale di manutenzione, manuale di installazione, e documentazione API.
			\item \textbf{Demo}: presentazione del prodotto sviluppato dall'azienda. 
		\end{enumerate}
	\subsection{Suddivisione del carico di lavoro}
		\begin{center}
			\begin{table}[h!]
				\centering
				\begin{tabular}{c | c | c | c} 
					\textbf{Fase} & \textbf{Data inizio} & \textbf{Data fine} & \textbf{Durata}\\
					\hline
					Conoscenze generali   & 26/04/2021 & 04/05/2021 & 64 \\
					Analisi dei requisiti       & 05/05/2021 & 08/05/2021 & 32 \\
					Progettazione tecnica & 10/05/2021 & 14/05/2021 & 40 \\
					Codifica                       & 15/05/2021 & 04/06/2021 & 144 \\
					Documentazione         & 05/06/2021 & 09/06/2021 & 32 \\
					Demo                           & 10/06/2021 & 10/06/2021 & 8 \\
					\hline\hline
					\multicolumn{3}{l}{\textbf{Totale}} & 320 \\
				\end{tabular}
				\vspace{0.3cm}
				\caption{Tabella della suddivisione delle ore per ogni fase del progetto di stage}
			\end{table}
		
		\end{center}
		
        %%******************************************%%
        %%                                          %%
        %%        Modello di tesi di laurea         %%
        %%            di Andrea Giraldin            %%
        %%                                          %%
        %%             2 novembre 2012              %%
        %%                                          %%
        %%******************************************%%


% I seguenti commenti speciali impostano:
% 1. 
% 2. PDFLaTeX come motore di composizione;
% 3. tesi.tex come documento principale;
% 4. il controllo ortografico italiano per l'editor.

% !TEX encoding = UTF-8
% !TEX TS-program = pdflatex
% !TEX root = tesi.tex
% !TEX spellcheck = it-IT

% PDF/A filecontents 
\RequirePackage{filecontents}
\begin{filecontents*}{\jobname.xmpdata}
  \Title{Realizzazione di un engine di monitoraggio delle segnalazioni per la notifica di violazioni dei service level agreement}
  \Author{Sinigaglia Alberto}
  \Language{it-IT}
  \Subject{Realizzazione di un engine di monitoraggio delle segnalazioni per la notifica di violazioni dei service level agreement}
  \Keywords{Service Level Agreement\sep Redmine\sep Quality Assurence\sep Rest API \sep Notification System}
\end{filecontents*}

\documentclass[10pt,                    % corpo del font principale
               a4paper,                 % carta A4
               twoside,                 % impagina per fronte-retro
               openright,               % inizio capitoli a destra
               english,                 
               italian,                 
               ]{book}    

%**************************************************************
% Importazione package
%************************************************************** 
\usepackage[table, dvipsnames]{xcolor}
\definecolor{red}{rgb}{0.6,0,0}
\definecolor{blue}{rgb}{0,0,0.6}
\definecolor{green}{rgb}{0,0.8,0}
\definecolor{cyan}{rgb}{0.0,0.6,0.6}
\definecolor{footer-gray}{HTML}{808080}
\definecolor{light-gray}{gray}{0.6} 
\definecolor{light-grayer}{gray}{0.75} 
\definecolor{lighter-grayer}{gray}{0.85} 
\definecolor{lightest-grayest}{gray}{0.94} 
\definecolor{codegreen}{rgb}{0,0.4,0.2}
\definecolor{codegray}{rgb}{0.5,0.5,0.5}
\definecolor{codepurple}{rgb}{0.58,0,0.82}
\definecolor{backcolour}{rgb}{0.95,0.95,0.96}


\PassOptionsToPackage{dvipsnames}{xcolor} % colori PDF/A

\usepackage{colorprofiles}

\usepackage[a-2b,mathxmp]{pdfx}[2018/12/22]
                                        % configurazione PDF/A
                                        % validare in https://www.pdf-online.com/osa/validate.aspx

%\usepackage{amsmath,amssymb,amsthm}    % matematica

\usepackage[T1]{fontenc}                % codifica dei font:
                                        % NOTA BENE! richiede una distribuzione *completa* di LaTeX

\usepackage[utf8]{inputenc}             % codifica di input; anche [latin1] va bene
                                        % NOTA BENE! va accordata con le preferenze dell'editor

\usepackage[english, italian]{babel}    % per scrivere in italiano e in inglese;
                                        % l'ultima lingua (l'italiano) risulta predefinita

\usepackage{bookmark}                   % segnalibri

\usepackage{caption}                    % didascalie

\usepackage{chngpage,calc}              % centra il frontespizio

\usepackage{csquotes}                   % gestisce automaticamente i caratteri (")

\usepackage{emptypage}                  % pagine vuote senza testatina e piede di pagina

\usepackage{epigraph}			% per epigrafi

\usepackage{eurosym}                    % simbolo dell'euro

%\usepackage{indentfirst}               % rientra il primo paragrafo di ogni sezione

\usepackage{graphicx}                   % immagini

\usepackage{hyperref}                   % collegamenti ipertestuali

\usepackage[binding=5mm]{layaureo}      % margini ottimizzati per l'A4; rilegatura di 5 mm

\usepackage{listings}                   % codici

\usepackage{microtype}                  % microtipografia

\usepackage{mparhack,fixltx2e,relsize}  % finezze tipografiche

\usepackage{nameref}                    % visualizza nome dei riferimenti                                      
\usepackage[font=small]{quoting}        % citazioni

\usepackage{subfig}                     % sottofigure, sottotabelle

\usepackage[italian]{varioref}          % riferimenti completi della pagina

\usepackage{booktabs}                   % tabelle                                       
\usepackage{tabularx}                   % tabelle di larghezza prefissata                                    
\usepackage{longtable}                  % tabelle su più pagine                                        
\usepackage{ltxtable}                   % tabelle su più pagine e adattabili in larghezza

\usepackage[toc, acronym]{glossaries}   % glossario
                                        % per includerlo nel documento bisogna:
                                        % 1. compilare una prima volta tesi.tex;
                                        % 2. eseguire: makeindex -s tesi.ist -t tesi.glg -o tesi.gls tesi.glo
                                        % 3. eseguire: makeindex -s tesi.ist -t tesi.alg -o tesi.acr tesi.acn
                                        % 4. compilare due volte tesi.tex.

\usepackage[backend=biber,style=verbose-ibid,hyperref,backref]{biblatex}
                                        % eccellente pacchetto per la bibliografia; 
                                        % produce uno stile di citazione autore-anno; 
                                        % lo stile "numeric-comp" produce riferimenti numerici
                                        % per includerlo nel documento bisogna:
                                        % 1. compilare una prima volta tesi.tex;
                                        % 2. eseguire: biber tesi
                                        % 3. compilare ancora tesi.tex.

\usepackage{multirow,tabularx}
\usepackage{geometry}
\geometry{
	margin=2.0in,
	top=30mm, % NON TOCCARE
	bottom=30mm,
	left=30mm,
	right=35mm
}


\input{tesi-config}                     % file con le impostazioni personali
\newcommand{\gloxy}[1]{\emph{#1}$_G$}
\setlength{\tabcolsep}{10pt}
\renewcommand{\arraystretch}{1.4}
\usepackage{hyperref}
\hypersetup{
	colorlinks,
	citecolor=black,
	filecolor=black,
	linkcolor=black,
	urlcolor=black
}


\begin{document}
%**************************************************************
% Materiale iniziale
%**************************************************************
\frontmatter
\input{inizio-fine/frontespizio}
\input{inizio-fine/colophon}
\input{inizio-fine/dedica}
% !TEX encoding = UTF-8
% !TEX TS-program = pdflatex
% !TEX root = ../tesi.tex

%**************************************************************
% Sommario
%**************************************************************
\cleardoublepage
\phantomsection
\pdfbookmark{Sommario}{Sommario}
\begingroup
\let\clearpage\relax
\let\cleardoublepage\relax
\let\cleardoublepage\relax

\chapter*{Sommario}

Il presente documento descrive il lavoro svolto durante il periodo di stage, della durata di circa trecento ore, dal laureando Alberto Sinigaglia presso l'azienda EsignWorld.\\
Lo stage aveva come obiettivo l'analisi e lo sviluppo di un sistema di monitoraggio delle segnalazioni, per la verifica e la notifica di potenziali violazioni delle S.L.A. stipulate con i clienti.\\
Il sistema in questione aveva come requisiti:
\begin{enumerate}
	\item interfacciarsi con il portale per la segnalazione di problemi utilizzato in azienda (Redmine)
	\item esser modulare, così da permettere una migrazione ad altri sistemi di segnalazione
	\item notificare tramite diversi canali potenziali violazioni delle S.L.A.
\end{enumerate}
Per far ciò è stata eseguita inizialmente una fase di studio dei componenti, seguita dall'analisi dei requisiti, per poi concludersi nello sviluppo del prodotto, che in conclusione ha portato alla demo finale.

%\vfill
%
%\selectlanguage{english}
%\pdfbookmark{Abstract}{Abstract}
%\chapter*{Abstract}
%
%\selectlanguage{italian}

\endgroup			

\vfill


% !TEX encoding = UTF-8
% !TEX TS-program = pdflatex
% !TEX root = ../tesi.tex

%**************************************************************
% Ringraziamenti
%**************************************************************
\cleardoublepage
\phantomsection
\pdfbookmark{Ringraziamenti}{ringraziamenti}

\begin{flushright}{
	\slshape    
	``Not everything that counts can be counted, \\and not everything that can be counted counts.''} \\ 
	\medskip
    --- Albert Einstein
\end{flushright}


\bigskip

\begingroup
\let\clearpage\relax
%\let\cleardoublepage\relax
\let\cleardoublepage\relax

\chapter*{Ringraziamenti}

\noindent \textit{Innanzitutto, vorrei esprimere la mia gratitudine alla Professoressa Gaggi Ombretta, relatore della mia tesi, per l'aiuto e il sostegno fornitomi durante la stesura del lavoro.}\\

\noindent \textit{Desidero ringraziare con affetto i miei genitori e annessi compagni che mi hanno sempre sostenuto, appoggiando ogni mia decisione, fin dalla scelta del mio percorso di studi.} \\
	
\noindent \textit{Infine vorrei ringraziare i miei amici e la mia fidanzata, per essermi sempre stati vicini, per aver creduto in me ed avermi sempre sostenuto, soprattutto nei momenti difficili.} \\

\bigskip

\noindent\textit{\myLocation, \myTime}
\hfill \myName

\endgroup


\input{inizio-fine/indici}
\cleardoublepage

%**************************************************************
% Materiale principale
%**************************************************************
\mainmatter
% !TEX encoding = UTF-8
% !TEX TS-program = pdflatex
% !TEX root = ../tesi.tex

%**************************************************************
\chapter{Introduzione}
\label{cap:introduzione}
%**************************************************************

Il seguente capitolo vuole introdurre brevemente il progetto affrontato e l'azienda ospitante ideatrice di esso. \\



\iffalse
\noindent Esempio di utilizzo di un termine nel glossario \\
\gls{api}. \\
\noindent Esempio di citazione in linea \\
\cite{site:agile-manifesto}. \\
\noindent Esempio di citazione nel pie' di pagina \\
citazione\footcite{womak:lean-thinking} \\
\fi


%**************************************************************
\section{L'azienda}

\par EsignWorld è un'azienda italiana facente parte del gruppo Euronovate, gruppo del settore IT che conta circa 120 collaboratori, con sede centrale in Svizzera con varie sedi in Spagna, Italia, Romania e Cina. 
\newline
\newline
\par EsignWorld si occupa di sviluppare e progettare soluzioni complesse nell'ambito della digital identity, digital signature e digital onboarding, sia software che hardware riguardanti i processi di dematerializzazione all'interno delle aziende.
\newline
\newline
\par Durante lo stage ci si è andati ad interfacciare con Redmine, che per quanto non sia un prodotto sviluppato direttamente da Euronovate, è utilizzato internamente per la pianificazione di progetti e per il tracciamento delle segnalazioni di bug.
\newline
\begin{figure}[!h] 
	\centering 
	\includegraphics[width=0.5\columnwidth]{logo-azienda.png} 
	\caption{Logo dell'azienda}
\end{figure}
\begin{figure}[!h] 
	\centering 
	\includegraphics[width=0.5\columnwidth]{logo-redmine.png} 
	\caption{Logo di Redmine}
\end{figure}
\newpage
%**************************************************************
\section{L'idea}

Lo stage consiste nello sviluppo di un prodotto software per il monitoraggio delle segnalazioni, con conseguente notifica in caso di violazioni dei service level agreement stipulati tra l'azienda e i clienti.\\
Per raggiungere tale obiettivo, ci si è appoggiati a Redmine, software adottato dall'azienda per la gestione delle segnalazioni e l'ottenimento dei dati necessari alle analisi. Redmine infatti raccoglie tutti i dati fondamentali, quali segnalazioni, tempo dedicato per ogni task e un log di tutti i cambiamenti avvenuti. \\
Il prodotto periodicamente quindi scarica gli aggiornamenti da Redmine, li analizza, e tramite delle policy predefinite per ogni cliente, va a notificare tramite vari canali l'eventuale violazione di esse. \\
Infine, l'applicativo espone un'API per l'ottenimento di informazioni statistiche riguardanti le violazioni. 

%**************************************************************
\section{Organizzazione del testo}

\begin{description}
    \item[{\hyperref[cap:descrizione-stage]{Il secondo capitolo}}] espone una panoramica sugli obbiettivi dello stage e la sua pianificazione, con un'analisi preventiva dei rischi.
    
    \item[{\hyperref[cap:analisi-requisiti]{Il terzo capitolo}}] descrive l'analisi dei requisiti del progetto affrontato nello stage, durante la quale son state definite le funzionalità desiderate.
    
    \item[{\hyperref[cap:progettazione-codifica]{Il quarto capitolo}}] approfondisce la progettazione e la codifica del progetto, esponendo successivamente le problematiche incontrate
    
    \item[{\hyperref[cap:verifica-validazione]{Il quinto capitolo}}] 
    espone tutte le verifiche effettaute sul progetto durante il suo sviluppo e la validazione finale, per garantire un buon risultato finale, conforme con i requisiti individuati
    
    \item[{\hyperref[cap:documentazione]{Il sesto capitolo}}] descive il processo di documentazione intrapreso durante lo stage
    
    \item[{\hyperref[cap:conclusioni]{Il settimo capitolo}}] espone le conclusioni tratte dallo stage, con annesse conoscenze acquisite e consuntivo finale
\end{description}

\noindent Riguardo la stesura del testo, relativamente al documento sono state adottate le seguenti convenzioni tipografiche:
\begin{itemize}
	\item gli acronimi, le abbreviazioni e i termini ambigui o di uso non comune menzionati vengono definiti nel glossario, situato alla fine del presente documento;
	\item per la prima occorrenza dei termini riportati nel glossario viene utilizzata la seguente nomenclatura: \gloxy{parola};
	\item i termini in lingua straniera o facenti parti del gergo tecnico sono evidenziati con il carattere \emph{corsivo}.
\end{itemize}             % Introduzione
\iffalse \input{capitoli/capitolo-2}      \fi       % Processi 
% !TEX encoding = UTF-8
% !TEX TS-program = pdflatex
% !TEX root = ../tesi.tex

%**************************************************************
\chapter{Descrizione dello stage}
\label{cap:descrizione-stage}
%**************************************************************

\intro{In questa sezione verrà trattato come si è svolto lo stage presso EsignWorld, considerando gli obiettivi pianificati, discutendo dei possibili rischi e la comunicazione con il tutor aziendale.}\\

\section{Introduzione}
Lo stage presso EsignWorld aveva come scopo la creazione di un engine che monitorasse i ticket aperti dai clienti dell'azienda per capire se si fosse in procinto di violare le Service Level Agreement stipulate con tale cliente. \\
Per far ciò, ci si è andati a interfacciare a Redmine, prodotto usato internamente dall'azienda per la gestione di progetti e l'apertura di ticket. Grazie ad esso quindi, ci si è potuti sollevare la parte di gestione del ticket e dei clienti, focalizzandosi solo sulla parte di analisi e notifica. \\
Una volta ottenuti i dati da Redmine, l'engine doveva appunto analizzare i dati, confrontarli con le S.L.A. fornitegli per ogni cliente, e in caso si fosse vicini al violarle, o la violazione fosse già avvenuta, procedere a segnalarlo al responsabile, tramite molteplici canali di notifica (quali per esempio Telegram e Email). \\
Infine, il progetto doveva predisporre anche un API per la pubblicazione dei dati: tale API aveva come scopo un monitoraggio statistico dei ticket, delle segnalazioni e delle violazioni, e quindi mirava a preparare i dati per esempio per graficarli o manipolarli.


%**************************************************************
\section{Obiettivi del progetto}
\begin{itemize}
	\item Principali:
	\begin{enumerate}
		\item Studio e valutazione di componenti esterni ai quali interfacciarsi successivamente;
		\item Test e recupero dati dai servizi esposti da Redmine;
		\item Analisi di vari sistemi di notifica, come Telegram e Email;
		\item Redazione di un documento di Analisi dei Requisiti;
		\item Redazione di un documento di Progettazione Tecnica;
		\item Sviluppo/Codifica del progetto seguendo le norme di codifica aziendali;
		\item Redazione documentazione del progetto:
		\begin{enumerate}
			\item Manuale di manutenzione;
			\item Manuale di installazione;
			\item Documentazione API.
		\end{enumerate}
	\end{enumerate}
	\item Secondari:
	\begin{enumerate}
		\item Integrazione dell'API con l'applicativo ENAnalytics;
		\item Integrazione della base di dati con l'applicativo Grafana;
		\item Sviluppo di test di unità e di integrità.
	\end{enumerate}
\end{itemize}


%**************************************************************
\section{Analisi preventiva dei rischi}

Durante la fase di analisi iniziale sono stati individuati alcuni possibili rischi a cui si poteva andare incontro.
Si è quindi proceduto a elaborare delle possibili soluzioni per far fronte ad essi.\\


\begin{risk}{Tecnologie nuove}
    \riskdescription{le tecnologie consigliate per lo sviluppo del progetto erano nuove, causando un'inevitabile inesperienza nel loro utilizzo}
    \risksolution{è stato predefinito un periodo iniziale di formazione personale per le tecnologie che son state poi utilizzate per lo sviluppo, così da aver tempo di familiarizzarci leggermente e arrivare alla codifica con le nozioni base già apprese}
\end{risk}
\begin{risk}{Sistemi esterni}
	\riskdescription{il progetto richiede di interfacciarsi con sistemi esterni (quali per esempio Redmine) mai usati prima e che potrebbero avere una documentazione non esaustiva}
	\risksolution{è stata predefinito un periodo iniziale di formazione personale sui sistemi ai quali ci si è interfacciati durante la codifica, così da avere già una chiara visione del loro funzionamento e delle feature che mettono a disposizione}
\end{risk}
\begin{risk}{Gestione dello smart working}
	\riskdescription{lo stage si è tenuto interamente da remoto, portando quindi a un potenziale rischio di mancanza di comunicazione, che avrebbe potuto portare a un'incerta comprensione degli obbiettivi dello stage e del progetto affrontato}
	\risksolution{si è deciso di avere un meeting giornaliero con il tutor aziendale nelle prime fasi del progetto, così da esser sicuri che si arrivasse alle fasi finali (come la codifica) con tutti i chiarimenti necessari espletati}
\end{risk}
 


%**************************************************************

\subsection{Pianificazione}
	Di seguito sono elencate le varie fasi previste per lo stage e una loro stima oraria, in ordine cronologico:
	\begin{enumerate}
		\item \textbf{Conoscenze generali}: ($\sim$1 giorno) approfondimento e installazione degli ambienti di sviluppo e di versionamento, e abilitazione strumenti aziendali;
		\item \textbf{Acquisizione degli standard aziendali}: ($\sim$1 giorno) esposizione dei servizi server Euronovate e dell'architettura della soluzione di firma Euronovate;
		\item \textbf{Formazione personale}: ($\sim$5 giorni) formazione sul framework target per lo sviluppo, valutazione e test di componenti esterni (sistemi di notifica come email e messaggistica istantanea), istruzione sull'applicativo Redmine e recupero dati dai suoi servizi esposti;
		\item \textbf{Analisi dei requisiti}: ($\sim$4 giorni) analisi del tipo di informazioni da monitorare e successivamente da esporre/pubblicare, selezione dei sistemi e delle regole di notifica (e relativa frequenza di aggiornamento dati locali);
		\item \textbf{Progettazione tecnica}: ($\sim$5 giorni) stesura della specifica tecnica del Back-End, dei servizi esposti e degli oggetti di scambio;
		\item \textbf{Codifica}: ($\sim$18 giorni) sviluppo del prodotto seguendo le norme di codifica aziendali e potenzialmente test di unità e integrità;
		\item \textbf{Documentazione}: ($\sim$4 giorni) stesura manuale di manutenzione, manuale di installazione, e documentazione API;
		\item \textbf{Demo}: ($\sim$1 giorno) presentazione del prodotto sviluppato dall'azienda. 
	\end{enumerate}
	Il tutto può essere visualizzato nel seguente Gantt Chart:
	\begin{center}
		\includegraphics[keepaspectratio = true, width=15cm]{immagini/pianificazione.png}
		\captionof{figure}{Pianificazione dello stage}
	\end{center}
\iffalse
	\subsection{Suddivisione del carico di lavoro}
		\begin{center}
			\begin{table}[h!]
				\centering
				\begin{tabular}{c | c | c | c} 
					\textbf{Fase} & \textbf{Data inizio} & \textbf{Data fine} & \textbf{Durata}\\
					\hline
					Conoscenze generali   & 26/04/2021 & 04/05/2021 & 64 \\
					Analisi dei requisiti       & 05/05/2021 & 08/05/2021 & 32 \\
					Progettazione tecnica & 10/05/2021 & 14/05/2021 & 40 \\
					Codifica                       & 15/05/2021 & 04/06/2021 & 144 \\
					Documentazione         & 05/06/2021 & 09/06/2021 & 32 \\
					Demo                           & 10/06/2021 & 10/06/2021 & 8 \\
					\hline\hline
					\multicolumn{3}{l}{\textbf{Totale}} & 320 \\
				\end{tabular}
				\vspace{0.3cm}
				\caption{Tabella della suddivisione delle ore per ogni fase del progetto di stage}
			\end{table}
		\end{center}
\fi
		             % Kick-Off
% !TEX encoding = UTF-8
% !TEX TS-program = pdflatex
% !TEX root = ../tesi.tex

%**************************************************************
\chapter{Analisi dei requisiti}
\label{cap:analisi-requisiti}
%**************************************************************

\intro{In questo capitolo viene esposta l'analisi dei requisiti effettutata durante lo stage, nella quale si descrivono le funzionalità e i requisiti identificati.}\\
\section{Analisi del prodotto}
	\subsection{Descrizione del prodotto}
		\subsubsection{Contesto}
			Il prodotto sarà utilizzato internamente dall'azienda Euronovate per il monitoraggio della validità delle \gloxy{S.L.A.}, e per un'analisi statistica delle segnalazioni aperte da \gloxy{Customers}.
		\subsubsection{Funzionalità} 
			Il prodotto deve interfacciarsi con il software \gloxy{Redmine}, già utilizzato dall'azienda per la raccolta delle segnalazioni da parte dei clienti, per l'ottenimento dei dati, i quali poi dovranno essere analizzati e in caso di violazioni, dovranno essere predisposti diversi canali di notifica, per la segnalazione della violazione.\\
			Infine, i dati saranno salvati in una base di dati, ed esposti tramite un \gloxy{API} con un fine di analisi statistica di essi.
		\subsection{Analisi della struttura} 
			Il prodotto prevede principalmente 4 sistemi: l'\gloxy{API} di pubblicazione, il sistema di ottenimento dei dati, il sistema di analisi, ed infine il sistema di notifica.
		\subsubsection{Sistema ottenimento dati} 
			Il sistema ottenimento dati sistema è responsabile dell'ottenimento dei dati da \gloxy{Redmine}, quali \gloxy{Customers} e \gloxy{Ticket} aperti da clienti. \\
			Questo sistema deve autenticarsi su \gloxy{Redmine}, ottenere i dati, e manipolarli in modo da prevedere una loro semplice integrazione con gli altri servizi.
			Tale download dei dati verrà effettuato in batch, quindi periodicamente l'engine andrà a scaricarsi gli aggiornamenti. \\
			deve infine esser possibile migrare a un servizio di segnalazione esterno diverso da \gloxy{Redmine}, senza andare a modificare gli altri sistemi.
		\subsubsection{Sistema analisi dati}
			Il sistema analisi dati è responsabile dell'analisi dei dati e delle \gloxy{S.L.A.}. definite con i vari clienti. \\
			È quindi stato necessario predisporre un sistema di configurazione delle \gloxy{S.L.A.}. dei clienti dinamico, che non richieda il riavvio del sistema. \\
			Inoltre, una volta analizzato dati e \gloxy{S.L.A.}., è responsabile  della comunicazione con il servizio di notifica per la segnalazione di violazioni.
		\subsubsection{Sistema di notifica}
			Il sistema di notifica sarà responsabile della notifica delle violazioni ai responsabili.\\
			È stato quindi necessario predisporre vari mezzi di notifica, configurabili dinamicamente, che permettano la segnalazione a chi di dovere della potenziale violazione delle \gloxy{S.L.A.}., in base a quanto grave essa sia.
		\subsubsection{Sistema di pubblicazione}
			Il sistema di pubblicazione è responsabile della pubblicazione dei dati gestiti verso l'esterno.\\
			È quindi stato necessario predisporre un \gloxy{API} tramite la quale gli interessati potranno accedere alle informazioni gestite dall'engine, quali issue, customer e violazioni. \\
			È desiderabile permettere di definire se il risultato lo si vuole dalla base di dati attuale (e quindi aggiornato all'ultimo update), o lo si vuole esattamente fino a quel momento, e quindi richiedendo un aggiornamento immediato da \gloxy{Redmine}.
		






\section{Casi d'uso}

In questa sezione sono elencati i casi d'uso rilevati nell'analisi dei requisiti del progetto.\\
Per lo studio dei casi di utilizzo del prodotto sono stati creati dei diagrammi.\\
I diagrammi dei casi d'uso (in inglese \emph{Use Case Diagram}) sono diagrammi di tipo \gls{uml} dedicati alla descrizione delle funzioni o servizi offerti da un sistema, così come sono percepiti e utilizzati dagli attori che interagiscono col sistema stesso.
Essendo il progetto incentrato sullo sviluppo dell'engine, le interazioni da parte dell'utilizzatore sono ovviamente ridotte allo stretto necessario. Per questo motivo i diagrammi d'uso risultano semplici e in numero ridotto.\\

\noindent Ogni caso d’uso ha un codice gerarchico ed univoco che lo identifica, nella forma:
\begin{center}
	\textbf{UC<CodicePadre>.<CodiceFiglio>}
\end{center}
Il codice progressivo può includere diversi livelli di gerarchia separati da un punto.

\subsection{Attori}
Gli attori individuati dopo un’attenta analisi sono i seguenti:


\begin{center}
	\includegraphics[keepaspectratio = true, width=15cm]{immagini/actors.png}
	\captionof{figure}{Attori individuati}
\end{center}

\subsubsection{Redmine}
\gloxy{Redmine} rappresenta il sistema di gestione di progetto e tracciamento delle segnalazioni utilizzato dall'azienda.
\subsubsection{Telegram}
\gloxy{Telegram} rappresenta uno dei canali di notifica identificati per la segnalazione delle violazioni.
\subsubsection{Server SMTP}
L'email rappresenta uno dei canali di notifica identificati per la segnalazione delle violazioni.
\subsubsection{Client}
Con Client si vuole identificare qualsiasi attore capace di consumare l'\gloxy{API}  (come un Front-end, un altro server ...).
\subsubsection{Internal System}
Con Internal System si vuole identificare la piattaforma ENTicketEngine stessa.


\subsection{Sistema di ottenimento dati}
Gli use case identificati per questo sistema possono essere riassunti mediante il seguente diagramma UML:
\begin{center}
	\includegraphics[keepaspectratio = true, width=15cm]{immagini/uc/1.png}
	\captionof{figure}{Use Case sistema di ottenimento dati}
\end{center}
\subsubsection{UC1 - Connessione a Redmine}
\begin{itemize}
	\item \textbf{attore primario}: Internal System;
	\item \textbf{attore secondario}: \gloxy{Redmine};
	\item \textbf{descrizione}: l'engine desidera connettersi a \gloxy{Redmine} per accedere alle sue funzionalità;
	\item \textbf{precondizione}: l'engine non ha ancora effettuato la connessione a \gloxy{Redmine};
	\item \textbf{postcondizione}: l'engine ha una connessione a \gloxy{Redmine} stabilita;
	\item \textbf{scenario principale}: 
	\begin{enumerate}
		\item Il sistema invia una richiesta di connessione a \gloxy{Redmine}.
	\end{enumerate}
\end{itemize}

\subsubsection{UC2 - Autenticazione su Redmine}
\begin{itemize}
	\item \textbf{attore primario}: Internal System;
	\item \textbf{attore secondario}: \gloxy{Redmine};
	\item \textbf{descrizione}: l'engine vuole autenticarsi presso \gloxy{Redmine} così da poter accedere alle sue funzionalità;
	\item \textbf{precondizione}: l'engine ha una connessione disponibile verso il server \gloxy{Redmine};
	\item \textbf{postcondizione}: il sistema è autenticato presso \gloxy{Redmine} e ha accesso ai dati;
	\item \textbf{scenario principale}: 
	\begin{enumerate}
		\item il sistema invia una richiesta a \gloxy{Redmine} contenente la propria \gloxy{API}  key per l'autenticazione.
	\end{enumerate}
\end{itemize}
\subsubsection{UC3 - Ottenimento informazioni da Redmine}
\begin{center}
	\includegraphics[keepaspectratio = true, width=15cm]{immagini/uc/2.png}
	\captionof{figure}{Sottocasi d'uso UC3}
\end{center}
\begin{itemize}
	\item \textbf{attore primario}: Internal System;
	\item \textbf{attore secondario}: \gloxy{Redmine};
	\item \textbf{descrizione}: l'engine vuole ottenere le informazioni da \gloxy{Redmine};
	\item \textbf{precondizione}: l'engine ha una connessione disponibile verso il server \gloxy{Redmine} ed è autenticato presso esso;
	\item \textbf{postcondizione}: il sistema ottiene i dati richiesti da \gloxy{Redmine};
	\item \textbf{scenario principale}: 
	\begin{enumerate}
		\item il sistema invia una richiesta a \gloxy{Redmine} verso l'endpoint stabilito contenente la propria \gloxy{API key} usata l'autenticazione;
		\item il sistema riceve i dati richiesti.
	\end{enumerate}
\end{itemize}
\paragraph{UC3.1 - Ottenimento Customer}
\begin{itemize}
	\item \textbf{attore primario}: Internal System;
	\item \textbf{attore secondario}: \gloxy{Redmine};
	\item \textbf{descrizione}: l'engine vuole ottenere i \gloxy{Customers}  da \gloxy{Redmine};
	\item \textbf{precondizione}: l'engine ha una connessione disponibile verso il server \gloxy{Redmine} ed è autenticato presso di esso;
	\item \textbf{postcondizione}: il sistema ottiene i \gloxy{Customers}  da \gloxy{Redmine};
	\item \textbf{scenario principale}: 
	\begin{enumerate}
		\item il sistema invia una richiesta a \gloxy{Redmine} verso l'endpoint stabilito contenente la propria \gloxy{API key} usata l'autenticazione;
		\item il sistema riceve la lista dei \gloxy{Customers}  richiesti.
	\end{enumerate}
\end{itemize}
\paragraph{UC3.2 - Ottenimento Ticket}
\begin{itemize}
	\item \textbf{attore primario}: Internal System;
	\item \textbf{attore secondario}: \gloxy{Redmine};
	\item \textbf{descrizione}: l'engine vuole ottenere i Ticket da \gloxy{Redmine};
	\item \textbf{precondizione}: l'engine ha una connessione disponibile verso il server \gloxy{Redmine} ed è autenticato presso di esso;
	\item \textbf{postcondizione}: il sistema ottiene i ticket da \gloxy{Redmine};
	\item \textbf{scenario principale}: 
	\begin{enumerate}
		\item il sistema invia una richiesta a \gloxy{Redmine} verso l'endpoint stabilito contenente la propria \gloxy{API key} usata l'autenticazione;
		\item il sistema riceve la lista dei Ticket richiesti.
	\end{enumerate}
\end{itemize}
\subsection{Sistema di analisi dati}
Gli use case identificati per questo sistema possono essere riassunti mediale il seguente diagramma UML:
\begin{center}
	\includegraphics[keepaspectratio = true, width=15cm]{immagini/uc/3.png}
		\captionof{figure}{Use Case sistema analisi dati}
\end{center}
\subsubsection{UC4 - Configurazione dinamica dei Customers}
\begin{center}
	\includegraphics[keepaspectratio = true, width=15cm]{immagini/uc/4.png}
		\captionof{figure}{Sottocasi d'uso UC4}
\end{center}
\begin{itemize}
	\item \textbf{attore primario}: Internal System;
	\item \textbf{descrizione}: il sistema deve disporre di un metodo di configurazione dinamica dei vati \gloxy{Customers} ;
	\item \textbf{precondizione}: il sistema è in uno stato pronto per leggere la configurazione;
	\item \textbf{postcondizione}: il sistema ha letto la configurazione correttamente;
	\item \textbf{scenario principale}: 
	\begin{enumerate}
		\item il sistema apre un file di configurazione e legge le informazioni necessarie.
	\end{enumerate}
\end{itemize}
\paragraph{UC4.1 - Configurazione del tempo massimo di presa in carico di un Ticket}
\begin{itemize}
	\item \textbf{attore primario}: Internal System;
	\item \textbf{descrizione}: il sistema deve disporre di un modo per configurare il tempo massimo di presa in carico di un ticket per ogni \gloxy{Customer}; 
	\item \textbf{precondizione}: il sistema è pronto per leggere la propria configurazione;
	\item \textbf{postcondizione}: il sistema ha letto la configurazione correttamente per ogni \gloxy{Customer}; 
	\item \textbf{scenario principale}: 
	\begin{enumerate}
		\item il sistema legge la configurazione e viene a conoscenza del tempo massimo di presa in carico di un Ticket per ogni suo Customer.
	\end{enumerate}
\end{itemize}
\paragraph{UC4.2 - Configurazione del tempo massimo di presa in risoluzione di un Ticket}
\begin{itemize}
	\item \textbf{attore primario}: Internal System;
	\item \textbf{descrizione}: il sistema deve disporre di un modo per configurare il tempo massimo di risoluzione di un ticket per ogni \gloxy{Customer}; 
	\item \textbf{precondizione}: il sistema è pronto per leggere la propria configurazione;
	\item \textbf{postcondizione}: il sistema ha letto la configurazione correttamente per ogni \gloxy{Customer}; 
	\item \textbf{scenario principale}: 
	\begin{enumerate}
		\item il sistema legge la configurazione e viene a conoscenza del tempo massimo di risoluzione di un Ticket per ogni suo Customer.
	\end{enumerate}
\end{itemize}
\paragraph{UC4.3 - Configurazione preavviso di violazione delle S.L.A.}
\begin{itemize}
	\item \textbf{attore primario}: Internal System;
	\item \textbf{descrizione}: il sistema deve disporre di un modo per configurare il preavviso di notifica che si vuole avere prima di violare una \gloxy{S.L.A.};
	\item \textbf{precondizione}: il sistema è pronto per leggere la propria configurazione;
	\item \textbf{postcondizione}: il sistema ha letto la configurazione correttamente per ogni \gloxy{Customer}; 
	\item \textbf{scenario principale}: 
	\begin{enumerate}
		\item il sistema legge la configurazione e viene a conoscenza del preavviso di notifica desiderato per ogni suo Customer.
	\end{enumerate}
\end{itemize}

\subsubsection{UC5 - Verifica delle violazioni}
\begin{itemize}
	\item \textbf{attore primario}: Internal System;
	\item \textbf{descrizione}: il sistema deve poter analizzare i dati basandoli sulle configurazioni;
	\item \textbf{precondizione}: il sistema è in uno stato pronto per leggere la configurazione;
	\item \textbf{postcondizione}: il sistema ha letto la configurazione correttamente;
	\item \textbf{scenario principale}: 
	\begin{enumerate}
		\item il sistema apre un file di configurazione e legge le informazioni necessarie.
	\end{enumerate}
\end{itemize}
\subsubsection{UC6 - Salvataggio dati per analisi future}
\begin{itemize}
	\item \textbf{attore primario}: Internal System;
	\item \textbf{descrizione}: il sistema deve poter salvare i dati per analisi future;
	\item \textbf{precondizione}: il sistema ha ricevuto dei dati non ancora presenti nella base di dati;
	\item \textbf{postcondizione}: il sistema ha una base di dati aggiornata con i dati appena ricevuti;
	\item \textbf{scenario principale}: 
	\begin{enumerate}
		\item il sistema riceve dei nuovi dati (e.g. da \gloxy{Redmine});
		\item il sistema salva i dati mancanti nella base di dati.
	\end{enumerate}
\end{itemize}
\subsection{Sistema di notifica}
Gli use case identificati per questo sistema possono essere riassunti mediale il seguente diagramma UML:
\begin{center}
	\includegraphics[keepaspectratio = true, width=15cm]{immagini/uc/5.png}
		\captionof{figure}{Use Case sistema di notifica}
\end{center}
\subsubsection{UC7 - Invio notifica}
\begin{itemize}
	\item \textbf{attore primario}: Internal System;
	\item \textbf{descrizione}: il sistema deve notificare una potenziale violazione delle \gloxy{S.L.A.};
	\item \textbf{precondizione}: il sistema ha trovato una potenziale violazione delle \gloxy{S.L.A.};
	\item \textbf{postcondizione}: il sistema ha notificato chi di dovere della potenziale violazione;
	\item \textbf{scenario principale}: 
	\begin{enumerate}
		\item il sistema viene avvisato di una potenziale violazione;
		\item il sistema invia la notifica al responsabile.
	\end{enumerate}
\end{itemize}
\paragraph{UC7.1 -  Notifica tramite Telegram}
\begin{itemize}
	\item \textbf{attore primario}: Internal System;
	\item \textbf{descrizione}: il sistema deve notificare una potenziale violazione delle \gloxy{S.L.A.};
	\item \textbf{precondizione}: il sistema ha trovato una potenziale violazione delle \gloxy{S.L.A.} e il \gloxy{Customer} associato ha \gloxy{Telegram} come canale di notifica;
	\item \textbf{postcondizione}: il sistema ha notificato chi di dovere tramite \gloxy{Telegram}  della potenziale violazione;
	\item \textbf{scenario principale}: 
	\begin{enumerate}
		\item il sistema viene avvisato di una potenziale violazione;
		\item il sistema invia la notifica tramite \gloxy{Telegram} al responsabile.
	\end{enumerate}
\end{itemize}
\paragraph{UC7.2 -  Notifica tramite email}
\begin{itemize}
	\item \textbf{attore primario}: Internal System;
	\item \textbf{descrizione}: il sistema deve notificare una potenziale violazione delle \gloxy{S.L.A.};
	\item \textbf{precondizione}: il sistema ha trovato una potenziale violazione delle \gloxy{S.L.A.} e il \gloxy{Customer} associato ha Email come canale di notifica;
	\item \textbf{postcondizione}: il sistema ha notificato chi di dovere tramite Email  della potenziale violazione;
	\item \textbf{scenario principale}: 
	\begin{enumerate}
		\item il sistema viene avvisato di una potenziale violazione;
		\item il sistema invia la notifica tramite Email al responsabile.
	\end{enumerate}
\end{itemize}
\subsubsection{UC8 - Gestione configurazione client di notifica}
\begin{itemize}
	\item \textbf{attore primario}: Internal System;
	\item \textbf{descrizione}: il sistema deve rendere disponibile un mezzo di configurazione per i vari canali di notifica;
	\item \textbf{precondizione}: il sistema è pronto per leggere la propria configurazione;
	\item \textbf{postcondizione}: il sistema ha letto correttamente la configurazione per ogni canale di notifica;
	\item \textbf{scenario principale}: 
	\begin{enumerate}
		\item il sistema viene avviato;
		\item il sistema legge la configurazione di ogni canale di notifica (e.g. le credenziali per \gloxy{Telegram}).
	\end{enumerate}
\end{itemize}
\subsubsection{UC9 - Gestione configurazione notifica}

\begin{center}
	\includegraphics[keepaspectratio = true, width=15cm]{immagini/uc/6.png}
		\captionof{figure}{Sottocasi d'uso UC9}
\end{center}
\begin{itemize}
	\item \textbf{attore primario}: Internal System;
	\item \textbf{descrizione}: il sistema deve rendere disponibile un mezzo di configurazione per specificare il canale di notifica appropriato;
	\item \textbf{precondizione}: il sistema è pronto per leggere la propria configurazione;
	\item \textbf{postcondizione}: il sistema ha letto correttamente la configurazione per ogni \gloxy{Customer}; 
	\item \textbf{scenario principale}: 
	\begin{enumerate}
		\item il sistema viene avviato;
		\item il sistema legge la configurazione con le preferenze di notifica di ogni Customer.
	\end{enumerate}
\end{itemize}
\paragraph{UC9.1 -  Gestione canali di notifica per Customer}
\begin{itemize}
	\item \textbf{attore primario}: Internal System;
	\item \textbf{descrizione}: iIl sistema deve permettere di specificare che canali usare per la notifica di ogni \gloxy{Customer}; 
	\item \textbf{precondizione}: il sistema ha trovato una potenziale violazione delle \gloxy{S.L.A.} e il \gloxy{Customer} proprietario ha dei canali di notifica associati;
	\item \textbf{postcondizione}: il sistema ha notificato chi di dovere tramite i canali specificati della potenziale violazione;
	\item \textbf{scenario principale}: 
	\begin{enumerate}
		\item il sistema viene avvisato di una potenziale violazione.
		\item il sistema invia la notifica tramite i canali specificati avvisando di tale violazione.
	\end{enumerate}
\end{itemize}
\paragraph{UC9.2 -  Gestione cadenza di notifica per Customer}
\begin{itemize}
	\item \textbf{attore primario}: Internal System;
	\item \textbf{descrizione}: iIl sistema deve permettere di specificare che cadenza usare per la notifica di ogni \gloxy{Customer}; 
	\item \textbf{precondizione}: il sistema ha trovato una potenziale violazione delle \gloxy{S.L.A.} e il \gloxy{Customer} proprietario ha una cadenza di notifica associata;
	\item \textbf{postcondizione}: il sistema ha verificato se la cadenza è valida e in caso inviato la notifica;
	\item \textbf{scenario principale}: 
	\begin{enumerate}
		\item il sistema viene avvisato di una potenziale violazione;
		\item il sistema verifica se la cadenza è avvalorata, e se lo è invia la notifica di tale violazione.
	\end{enumerate}
\end{itemize}
\paragraph{UC9.3 -  Gestione urgenza di notifica per Customer}
\begin{itemize}
	\item \textbf{attore primario}: Internal System;
	\item \textbf{descrizione}: iIl sistema deve permettere di specificare che urgenza usare per la notifica di ogni \gloxy{Customer}; 
	\item \textbf{precondizione}: il sistema ha trovato una potenziale violazione delle \gloxy{S.L.A.} e il \gloxy{Customer} proprietario ha un'urgenza associata;
	\item \textbf{postcondizione}: il sistema ha inviato la notifica con l'urgenza associata;
	\item \textbf{scenario principale}: 
	\begin{enumerate}
		\item il sistema viene avvisato di una potenziale violazione;
		\item il sistema invia la notifica di violazione con l'urgenza associata a quel Customer.
	\end{enumerate}
\end{itemize}
\paragraph{UC9.4 -  Gestione gruppi telegram da usare}
\begin{itemize}
	\item \textbf{attore primario}: Internal System;
	\item \textbf{descrizione}: il sistema deve permettere di specificare che gruppi \gloxy{Telegram} usare per la notifica di ogni \gloxy{Customer}; 
	\item \textbf{precondizione}: il sistema ha trovato una potenziale violazione delle \gloxy{S.L.A.} e il \gloxy{Customer} proprietario ha dei gruppi \gloxy{Telegram} associati;
	\item \textbf{postcondizione}: il sistema ha inviato la notifica ai gruppi specificati;
	\item \textbf{scenario principale}: 
	\begin{enumerate}
		\item il sistema viene avvisato di una potenziale violazione;
		\item il sistema invia la notifica di violazione ai gruppi specificati.
	\end{enumerate}
\end{itemize}
\paragraph{UC9.5 -  Gestione indirizzi email da notificare}
\begin{itemize}
	\item \textbf{attore primario}: Internal System;
	\item \textbf{descrizione}: iIl sistema deve permettere di specificare che indirizzi email usare per la notifica di ogni \gloxy{Customer}; 
	\item \textbf{precondizione}: il sistema ha trovato una potenziale violazione delle \gloxy{S.L.A.} e il \gloxy{Customer} proprietario ha degli indirizzi email associati;
	\item \textbf{postcondizione}: il sistema ha inviato la notifica agli indirizzi email specificati;
	\item \textbf{scenario principale}: 
	\begin{enumerate}
		\item il sistema viene avvisato di una potenziale violazione;
		\item il sistema invia la notifica di violazione agli indirizzi specificati.
	\end{enumerate}
\end{itemize}

\subsection{Sistema di pubblicazione}
Gli use case identificati per questo sistema possono essere riassunti mediale il seguente diagramma UML:
\begin{center}
	\includegraphics[keepaspectratio = true, width=15cm]{immagini/uc/7.png}
		\captionof{figure}{Use Case sistema di notifica}
\end{center}
\subsubsection{UC10 - Ottenimento issue}
\begin{itemize}
	\item \textbf{attore primario}: Client;
	\item \textbf{descrizione}: il sistema deve permettere ad un client di ottenere le \gloxy{issue};
	\item \textbf{precondizione}: il sistema è pronto per ricevere richieste;
	\item \textbf{postcondizione}: il client riceve le \gloxy{issue} richieste;
	\item \textbf{scenario principale}: 
	\begin{enumerate}
		\item il client invia una richiesta di ottenimento \gloxy{issue};
		\item il server risponde con la lista delle \gloxy{issue} richieste.
	\end{enumerate}
\end{itemize}
\subsubsection{UC11 - Filtraggio issue}
\begin{center}
	\includegraphics[keepaspectratio = true, width=15cm]{immagini/uc/8.png}
		\captionof{figure}{Sottocasi d'uso UC11}
\end{center}
\begin{itemize}
	\item \textbf{attore primario}: Client;
	\item \textbf{descrizione}: il client vuole ottenere le \gloxy{issue} filtrate rispetto alcuni criteri;
	\item \textbf{precondizione}: il sistema è pronto per ricevere richieste;
	\item \textbf{postcondizione}:  il client riceve le \gloxy{issue} filtrate rispetto ai parametri inviati;
	\item \textbf{scenario principale}: 
	\begin{enumerate}
		\item il client invia una richiesta di ottenimento \gloxy{issue};
		\item il server risponde con la lista delle \gloxy{issue} richieste filtrare rispetto ai parametri inviati.
	\end{enumerate}
\end{itemize}
\paragraph{UC11.1 - Filtro per data minima}
\begin{itemize}
	\item \textbf{attore primario}: Client;
	\item \textbf{descrizione}: iIl sistema deve permettere di specificare la data minima per il filtraggio delle \gloxy{issue};
	\item \textbf{precondizione}:  il sistema è pronto per ricevere richieste;
	\item \textbf{postcondizione}: il client riceve le \gloxy{issue} filtrate rispetto alla data di apertura (successiva alla data specificata nella richiesta);
	\item \textbf{scenario principale}: 
	\begin{enumerate}
		\item  il client invia una richiesta di ottenimento \gloxy{issue} specificando una data di apertura minima;
		\item  il server risponde con la lista delle \gloxy{issue} richieste filtrate rispetto alla data di apertura (maggiore di quella richiesta).
	\end{enumerate}
\end{itemize}
\paragraph{UC11.2 - Filtro per data massima}
\begin{itemize}
	\item \textbf{attore primario}: Client;
	\item \textbf{descrizione}: iIl sistema deve permettere di specificare la data massima per il filtraggio delle \gloxy{issue};
	\item \textbf{precondizione}:  il sistema è pronto per ricevere richieste;
	\item \textbf{postcondizione}: il client riceve le \gloxy{issue} filtrate rispetto alla data di apertura (precedente alla data specificata nella richiesta);
	\item \textbf{scenario principale}: 
	\begin{enumerate}
		\item  il client invia una richiesta di ottenimento \gloxy{issue} specificando una data di apertura massima;
		\item  il server risponde con la lista delle \gloxy{issue} richieste filtrate rispetto alla data di apertura (minore di quella richiesta).
	\end{enumerate}
\end{itemize}
\paragraph{UC11.3 - Filtro per Customer}
\begin{itemize}
	\item \textbf{attore primario}: Client;
	\item \textbf{descrizione}: iIl sistema deve permettere di specificare il \gloxy{Customer} di cui si vuole ottenere le \gloxy{issue};
	\item \textbf{precondizione}:  il sistema è pronto per ricevere richieste;
	\item \textbf{postcondizione}: il client riceve le \gloxy{issue} filtrate rispetto al \gloxy{Customer} richiesto (precedente alla data specificata nella richiesta);
	\item \textbf{scenario principale}: 
	\begin{enumerate}
		\item  il client invia una richiesta di ottenimento \gloxy{issue} specificando un \gloxy{Customer}; 
		\item  il server risponde con la lista delle \gloxy{issue} richieste filtrate rispetto al \gloxy{Customer} richiesto.
	\end{enumerate}
\end{itemize}

\subsubsection{UC12 - Ottenimento violazioni}
\begin{itemize}
	\item \textbf{attore primario}: Client;
	\item \textbf{descrizione}: il client vuole ottenere una lista di violazioni;
	\item \textbf{precondizione}:  il sistema è pronto per ricevere richieste;
	\item \textbf{postcondizione}: il client riceve la lista delle violazioni richiesta;
	\item \textbf{scenario principale}: 
	\begin{enumerate}
		\item il client invia una richiesta di ottenimento delle violazioni;
		\item il server risponde con la lista delle violazioni richieste.
	\end{enumerate}
\end{itemize}
\subsubsection{UC13 - Raggruppamento issue}
\begin{itemize}
	\item \textbf{attore primario}: Client;
	\item \textbf{descrizione}: il client vuole ottenere una lista di \gloxy{issue} raggruppate secondo un certo criterio;
	\item \textbf{precondizione}:  il sistema è pronto per ricevere richieste;
	\item \textbf{postcondizione}: il client riceve la lista delle \gloxy{issue} raggruppata come richiesta;
	\item \textbf{scenario principale}: 
	\begin{enumerate}
		\item il client invia una richiesta di ottenimento delle \gloxy{issue} specificando un criterio di raggruppamento;
		\item il server risponde con la lista delle \gloxy{issue} raggruppata come richiesta.
	\end{enumerate}
\end{itemize}
\paragraph{UC13.1 - Raggruppamento issue per giorno}
\begin{itemize}
	\item \textbf{attore primario}: Client;
	\item \textbf{descrizione}: il client vuole ottenere una lista di \gloxy{issue} raggruppate per giorno di apertura;
	\item \textbf{precondizione}:  il sistema è pronto per ricevere richieste;
	\item \textbf{postcondizione}: il client riceve la lista delle \gloxy{issue} raggruppata come richiesta;
	\item \textbf{scenario principale}: 
	\begin{enumerate}
		\item il client invia una richiesta di ottenimento delle \gloxy{issue} richiedendo di raggrupparle per giorno;
		\item il server risponde con la lista delle \gloxy{issue} raggruppata per giorno.
	\end{enumerate}
\end{itemize}
\paragraph{UC13.1 - Raggruppamento issue per settimana}
\begin{itemize}
	\item \textbf{attore primario}: Client;
	\item \textbf{descrizione}: il client vuole ottenere una lista di \gloxy{issue} raggruppate per settimana di apertura;
	\item \textbf{precondizione}:  il sistema è pronto per ricevere richieste;
	\item \textbf{postcondizione}: il client riceve la lista delle \gloxy{issue} raggruppata come richiesta;
	\item \textbf{scenario principale}: 
	\begin{enumerate}
		\item il client invia una richiesta di ottenimento delle \gloxy{issue} richiedendo di raggrupparle per settimana;
		\item il server risponde con la lista delle \gloxy{issue} raggruppata per settimana.
	\end{enumerate}
\end{itemize}
\paragraph{UC13.1 - Raggruppamento issue per mese}
\begin{itemize}
	\item \textbf{attore primario}: Client;
	\item \textbf{descrizione}: il client vuole ottenere una lista di \gloxy{issue} raggruppate per mese di apertura;
	\item \textbf{precondizione}:  il sistema è pronto per ricevere richieste;
	\item \textbf{postcondizione}: il client riceve la lista delle \gloxy{issue} raggruppata come richiesta;
	\item \textbf{scenario principale}: 
	\begin{enumerate}
		\item il client invia una richiesta di ottenimento delle \gloxy{issue} richiedendo di raggrupparle per mese;
		\item il server risponde con la lista delle \gloxy{issue} raggruppata per mese.
	\end{enumerate}
\end{itemize}















\section{Tracciamento requisiti}
In questa sezione, vengono riportati i requisiti del progetto, classificati per obbligatorietà. Ciascun requisito possiede un codice identificativo, il cui formalismo viene riportato di seguito:
\begin{center}
	\textbf{R<NumeroRequisito>.<NumeroSottoRequisito>-<Classificazione>}
\end{center}
La classificazione andrà a specificare l'obbligatorietà del requisito, e potrà essere indicata tramite una delle due seguenti sigle:
\begin{itemize}
	\item \textbf{O}: requisito obbligatorio
	\item \textbf{D}: requisito desiderabile
\end{itemize}

\subsection{Requisiti}

\begin{center}
	\rowcolors{2}{lightest-grayest}{white}
	\begin{longtable}{|p{3cm}|p{8cm}|p{3cm}|}
		\hline
		\rowcolor{lighter-grayer}
		\textbf{Requisito} & \textbf{Descrizione} \\ \hline
		R1-O & L'engine deve interfacciarsi con \gloxy{Redmine} per l'ottenimento delle informazioni riguardanti a \gloxy{Customers}  e Issues \\ \hline
		R2-O & L'engine deve potersi autenticare presso \gloxy{Redmine} tramite \gloxy{API key} \\ \hline
		R3-O & L'engine deve permettere una configurazione dinamica delle \gloxy{S.L.A.} per ogni \gloxy{Customer} \\ \hline
		R3.1-O & L'engine deve permettere una configurazione dinamica della durata massima di presa in carico per i ticket di ogni \gloxy{Customer} \\ \hline
		R3.2-O & L'engine deve permettere una configurazione dinamica della durata massima di risoluzione per i ticket di ogni \gloxy{Customer} \\ \hline
		R4-O & L'engine deve rendere disponibili canali di notifica per la segnalazione di violazioni delle \gloxy{S.L.A.} \\ \hline
		R4.1-D & L'engine deve rendere disponibile Email come canale per la segnalazione di violazioni delle \gloxy{S.L.A.} \\ \hline
		R4.1-D & L'engine deve rendere disponibile la possibilità di specificare un'email come canale per la segnalazione di violazioni delle \gloxy{S.L.A.} customizzabile per ogni \gloxy{Customer}  \\ \hline
		R4.2-D & L'engine deve rendere disponibile \gloxy{Telegram}  come canale per la segnalazione di violazioni delle \gloxy{S.L.A.} \\ \hline
		R4.3-D & L'engine deve rendere disponibile  la possibilità di specificare un gruppo \gloxy{Telegram}  come canale per la segnalazione di violazioni delle \gloxy{S.L.A.} customizzabile per ogni \gloxy{Customer}  \\ \hline
		R5-O & L'engine deve notificare al responsabile designato potenziali violazioni delle \gloxy{S.L.A.} \\ \hline
		R6-D & L'engine deve permettere di configurare i vari canali di notifica definiti per la segnalazione di violazioni (e.g. le credenziali per la connessione SMTP)\\ \hline
		R7-O & L'engine deve mettere a disposizione un'\gloxy{API} per la pubblicazione dei dati\\ \hline
		R7.1-D & L'engine deve mettere a disposizione un'\gloxy{API} per la pubblicazione delle issue\\ \hline
		R7.2-D & L'engine deve mettere a disposizione un'\gloxy{API} per la pubblicazione delle violazioni\\ \hline
		R8-D & L'engine deve prevedere un modo per rendere personalizzabile l'intervallo di aggiornamento da \gloxy{Redmine}\\ \hline
		R9-O & L'engine deve permettere una facile migrazione a un sistema di gestione di Ticket diverso da \gloxy{Redmine}\\ \hline
		R10-O & L'engine deve prevedere degli \gloxy{S.L.A.} di default\\ \hline
		R11-O & L'engine deve prevedere la possibilità di categorizzare le violazioni per  “info”, “warning”, “urgent” ed “escalation”\\ \hline
		R12-O & L'engine deve prevedere la possibilità di aggiornare su richiesta i dati su \gloxy{Redmine} tramite \gloxy{API}\\ \hline
		R13-O & L'engine deve prevedere un servizio per esportare dei dati in qualche formato\\ \hline
		R14-O & L'engine deve prevedere un modo per memorizzare localmente gli aggiornamenti ricevuti da \gloxy{Redmine} \\ \hline
		R15-O & L'engine deve prevedere un modo per andare periodicamente a scaricare gli aggiornamenti da \gloxy{Redmine}\\ \hline
		
		\rowcolor{light-grayer}
		\multicolumn{2}{|l|}{\textbf{API}} \\ \hline
		R16-D & L'\gloxy{API} deve permettere il filtraggio tramite data minima dove possibile\\ \hline
		R17-D & L'\gloxy{API} deve permettere il filtraggio tramite data massima dove possibile\\ \hline
		R18-D & L'\gloxy{API} deve permettere di raggruppare i dati richiesti dove possibile\\ \hline
		R18.1-D & L'\gloxy{API} deve permettere di raggruppare i dati richiesti per giorno dove possibile\\ \hline
		R18.2-D & L'\gloxy{API} deve permettere di raggruppare i dati richiesti per settimana dove possibile\\ \hline
		R18.3-D & L'\gloxy{API} deve permettere di raggruppare i dati richiesti per mese dove possibile\\ \hline
		
		
		R19-D & L'\gloxy{API} deve fornire la possibilità di sapere per ogni \gloxy{Customer} quanti \gloxy{Ticket} ha avuto    \\ \hline
		R20-D & L'\gloxy{API} deve fornire la possibilità di sapere per ogni \gloxy{Customer} quante violazioni ha subito   \\ \hline
		R21-D & L'\gloxy{API} deve fornire la possibilità di sapere per ogni \gloxy{Ticket} quanto tempo in carico è stato a Euronovate e quanto al cliente\\ \hline
		R22-D & L'\gloxy{API} deve fornire la possibilità di ottenere il tempo medio di risoluzione di ticket\\ \hline
		R23-D & L'\gloxy{API} deve fornire la possibilità di ottenere il tempo medio di presa in carico di ticket\\ \hline
		
		
		
		% R-O & L'\gloxy{API}    \\ \hline
		
		\rowcolor{white}
		\caption{Elenco requisiti}
	\end{longtable}
\end{center}








             % Concept Preview
% !TEX encoding = UTF-8
% !TEX TS-program = pdflatex
% !TEX root = ../tesi.tex

%**************************************************************
\chapter{Progettazione e codifica}
\label{cap:progettazione-codifica}
%**************************************************************

\intro{In questo capitolo viene esposta la progettazione tecnica e la successiva codifica del prodotto ENTicketEngine.}\\
\section{Riferimenti}
    Di seguito vengono riportati i riferimenti ai vari tool e tecnologie previste per lo sviluppo del progetto:
    \begin{itemize}
        \item Java: \url{https://docs.oracle.com/en/java/javase/15/docs/api/index.html}
        \item API Redmine: \url{https://www.redmine.org/projects/redmine/wiki/rest_api}
        \item SDK Redmine: \url{https://github.com/taskadapter/redmine-java-api}
        \item Bot API Telegram: \url{https://core.telegram.org/bots}
        \item SDK Bot Telegram: \url{https://github.com/rubenlagus/TelegramBots}
        \item SDK Spring Mail: \url{https://mvnrepository.com/artifact/org.springframework.boot/spring-boot-starter-mail}
    \end{itemize}

\section{Analisi del prodotto}
	\subsection{Descrizione del prodotto}
		\subsubsection{Contesto}
			Il prodotto sarà utilizzato internamente dall'azienda Euronovate per il monitoraggio della validità delle \gloxy{S.L.A.}, e per un'analisi statistica delle segnalazioni aperte da \gloxy{Customers}.
			\subsubsection{Funzionalità}
			Il prodotto dovrà interfacciarsi con il software \gloxy{Redmine}, già utilizzato dall'azienda per la raccolta delle segnalazioni da parte dei clienti, per l'ottenimento dei dati, i quali poi dovranno essere analizzati e in caso di violazioni, saranno predisposti diversi canali di notifica, per la segnalazione della violazione.\\
			Infine, i dati saranno salvati in una base di dati, ed esposti tramite un \gloxy{API} con un fine di analisi statistica di essi.
			\newpage

\section{Analisi dell'architettura}
In questa sezione verrà analizzata l'architettura dei vari sistemi richiesti per la realizzazione del progetto\textbf{ENTicketEngine}.\\

    \subsection{Architettura}
        L'architettura del prodotto può essere riassunta con il seguente diagramma:
        \begin{center}
            \includegraphics[keepaspectratio = true, width=15cm]{immagini/progettazione/architettura.png}
            \captionof{figure}{Architettura generale del prodotto}
        \end{center}
        \subsection{Analisi dei sistemi}
            Il prodotto prevede principalmente 4 sistemi: l'\gloxy{API} di pubblicazione, il sistema di ottenimento dei dati, il sistema di analisi, ed infine il sistema di notifica.
        \subsubsection{Sistema ottenimento dati}
            Questo sistema è responsabile dell'ottenimento dei dati da \gloxy{Redmine}, quali \gloxy{Customers} e \gloxy{Ticket} aperti da clienti. \\
            Tale quindi doveva autenticarsi su \gloxy{Redmine}, ottenere i dati, e manipolarli in modo da prevedere una loro semplice integrazione con gli altri servizi.
            Tale download dei dati è stato effettuato in batch, quindi periodicamente l'engine è andato a scaricarsi gli aggiornamenti. \\
             Infine, bisognava predisporre questo sistema in modo da permettere di migrare a un servizio di segnalazione esterno diverso da \gloxy{Redmine}, senza andare a modificare gli altri sistemi.
            \begin{quote}
            	\mbox{}%
            	\vspace{-1cm}
                \paragraph{Dettaglio tecnico}
	                \mbox{}\\
                    Questo sistema essendo responsabile dell'ottenimento dei dati, si è dovuto appoggiare al servizio esterno scelto (in questo caso, \gloxy{Redmine}) per il raggiungimento del suo scopo. \\
                    Da Analisi dei Requisiti risulta un requisito che tale sistema però possa in futuro essere cambiato facilmente: ciò implica che il sistema di ottenimento dati debba essere il più possibile disaccoppiato dal sistema con il quale si andrà a interfacciare. \\
                    Per far ciò quindi, si è proceduto a creare un modulo per interfacciarsi al servizio esterno, che è stato accoppiato al sistema di ottenimento dati tramite \texttt{(Object) Adapter design pattern}.
                    \subparagraph*{Dipendenze esterne}
                    \mbox{} \\
                        Questo sistema si può scomporre in 2 sottosistemi:
                        \begin{enumerate}
                            \item un sottosistema per ottenere i dati;
                            \item un sottosistema per salvare gli oggetti ottenuti nella base di dati.
                        \end{enumerate}
                        Il tutto è rappresentato nel seguente grafico:
                        \begin{center}
                            \includegraphics[keepaspectratio = true, width=14cm]{immagini/progettazione/ottenimento.png}
                            \captionof{figure}{Struttura del sistema di ottenimento}
                        \end{center}
                        Come si può notare dal diagramma, fa uso della dipendenza \texttt{com.taskadapter:redmine-java-api} per andare a interfacciarsi con Redmine.
            \end{quote}
        \subsubsection{Sistema analisi dati}
            Questo sistema è responsabile dell'analisi dei dati e delle \gloxy{S.L.A.}. definite con i vari clienti. \\
            È stato quindi necessario predisporre un sistema di configurazione delle \gloxy{S.L.A.}. dei clienti dinamico, che non richiedesse il riavvio del sistema. \\
            Una volta analizzato dati e \gloxy{S.L.A.}., è responsabile  della comunicazione con il servizio di notifica per la segnalazione di violazioni.
            \begin{quote}
            	\mbox{}%
            	\vspace{-1cm}
                \paragraph{Dettaglio tecnico}
                	 \mbox{}\\
                    Questo sistema essendo responsabile della notifica delle violazioni, doveva prevedere degli algoritmi, che date le configurazioni lette inizialmente, analizzeranno i dati per trovare violazioni o segnalazioni. \\
                    Considerato che le necessità e i vincoli potevano cambiare, è stato usato \texttt{Strategy design pattern} per l'algoritmo, e considerato che le configurazioni delle \gloxy{S.L.A.} sono componibili (possono avere o meno una durata massima di presa in carico, possono o meno avere una durata massima di risoluzione, ecc ecc), si poteva far uso di \texttt{Decorator design pattern} per la logo generazione da file.
                    \subparagraph{Dipendenze esterne}
                    \mbox{} \\
                        Non sono state identificate dipendenze esterne per questo sistema.
            \end{quote}
        \subsubsection{Sistema di notifica}
            Questo sistema è responsabile della notifica delle violazioni ai responsabili.\\
            È stato quindi necessario predisporre vari mezzi di notifica, configurabili dinamicamente, che permettano la segnalazione a chi di dovere della potenziale violazione delle \gloxy{S.L.A.}., in base a quanto grave essa sia.
            \begin{quote}
            	\mbox{}%
            	\vspace{-1cm}
                \paragraph{Dettaglio tecnico}
                	 \mbox{}\\
                    Questo sistema, per raggiungere il suo obiettivo, doveva interfacciarsi con sistemi esterni, per i quali esistono già \gloxy{SDK} completi e testati. \\
                    Per permettere la modularità di tale sistema, e la possibilità di integrare nuovi canali di notifica senza andare a modificare codice pre-esistente, si è fatto uso del \texttt{Adapter design pattern} per conformare tutti i canali sotto un unica interfaccia, e nel caso tali SDK non fossero esistiti, si avrebbe fatto uso dell'\texttt{ereditarietà} per la loro creazione.
                    \subparagraph{Dipendenze esterne}
                    \mbox{} \\
                        Questo sistema si può scomporre in 2 sottosistemi:
                        \begin{enumerate}
                            \item un sottosistema per la gestione dei canali di notifica;
                            \item un sottosistema per ogni canale di notifica.
                        \end{enumerate}
                        Il tutto è rappresentato nel seguente grafico:
                        \begin{center}
                            \includegraphics[keepaspectratio = true, width=14cm]{immagini/progettazione/notifica.png}
                            \captionof{figure}{Struttura del sistema di notifica}
                        \end{center}
                        Come si può notare dal diagramma, fa uso delle seguenti dipendenze:
                        \begin{itemize}
                            \item \texttt{org.springframework.boot:spring-boot-starter-mail} per andare a interfacciarsi con Gmail per la notifica tramite mail;
                            \item \texttt{org.telegram:telegrambots} per andare a interfacciarsi con Telegram per la notifica tramite messaggio su gruppo.
                        \end{itemize} 
            \end{quote}
        \subsubsection{Sistema di pubblicazione}
            Questo sistema è responsabile della pubblicazione dei dati gestiti verso l'esterno.\\
            È stato quindi necessario predisporre un \gloxy{API} tramite la quale gli interessati potranno accedere alle informazioni gestite dall'engine, quali issue, customer e violazioni. \\
            \begin{quote}
            	\mbox{}%
            	\vspace{-1cm}
                \paragraph{Dettaglio tecnico}
                	 \mbox{}\\
                    Questo sistema, avente come responsabilità la pubblicazione di analisi statistiche sui dati, ha fatto uso del \texttt{MVC design pattern} per il suo sviluppo, così da disaccoppiare la logica tra modello, controller e viste, in modo da rendere modulare il sistema e permettere una potenziale futura migrazione a tipi di viste diverse (per esempio tramite XML) o l'aggiunta di nuovi endpoint.\\
                    \subparagraph{Dipendenze esterne}
                    \mbox{} \\
                        Questo sistema è stato, come precedentemente dichiarato, sviluppato seguendo il design pattern MVC, questo appoggiandosi al framework Java, Spring Boot. \\
                        Per questo motivo viene identificato \texttt{org.springframework.boot} come dipendenza esterna. 
            \end{quote}
        \newpage
    \subsection{Logica dell'engine}
        La logica che usa l'engine per raggiungere i suoi obbiettivi può essere riassunta con il seguente diagramma UML:
                \begin{center}
            \includegraphics[keepaspectratio = true, width=16.5cm]{immagini/progettazione/activity.png}
            \captionof{figure}{Diagramma di attività dell'algoritmo dell'engine}
        \end{center}
    
        
\section{Base di dati}
    Il progetto prevede l'uso di una base di dati per il mantenimento locale delle informazioni ottenute dal servizio esterno per la gestione di progetto. \\
    In particolare, questa base di dati ha l'obiettivo di mantenere le informazioni esposte dall'API successivamente descritta. \\
    \par La base di dati prevista è rappresentata dal seguente schema ER:
    \begin{center}
        \includegraphics[keepaspectratio = true, width=15cm]{immagini/progettazione/db.png}
        \captionof{figure}{Struttura della base di dati prevista}
    \end{center}
    \subsection{Analisi ridondanze}
        Lo schema ER visibile alla sezione precedente, presenta non minimalità, quali:
        \begin{itemize}
            \item \texttt{status} in \texttt{issues}, che potrebbe essere recuperato dallo stato del suo ultimo cambiamento;
            \item \texttt{time-company} e \texttt{time-customer} in \texttt{issues} che potrebbero essere ricalcolate considerando tutti le sue \texttt{changes};
            \item \texttt{open-issue} e \texttt{closed-issue} in \texttt{customers} che potrebbero essere ottenute contando le \texttt{issues} associate.
        \end{itemize} 
        Si è deciso di mantenere queste non-minimalità in modo da ottimizzare la pubblicazione di suddetti dati dall'API, in quanto focus primario del progetto.
    








\section{Endpoint API}
    \subsection{/customer/\{customer-id\}}
        \begin{itemize}
            \item \textbf{Tipo del metodo}: GET;
            \item \textbf{Descrizione}: endpoint per l'ottenimento delle informazioni rispetto a un customer.
            \item \textbf{Parametri}: \\
            \begin{center}
                \rowcolors{2}{lightest-grayest}{white}
                \begin{longtable}{|p{4cm}|p{4cm}|p{6cm}|}
                    \hline
                    \rowcolor{lighter-grayer}
                    \textbf{Nome} & \textbf{Tipo} & \textbf{Descrizione} \\ \hline
                    \texttt{min} & Timestamp GMT+2 & Data minima dell'apertura di ticket da considerare \\ \hline
                    \texttt{max} & Timestamp GMT+2 & Data massima dell'apertura di ticket da considerare \\ \hline
                \end{longtable}
            \end{center}
            \item \textbf{Risposte}: 
            \begin{center}
                \rowcolors{2}{lightest-grayest}{white}
                \begin{longtable}{|p{2.5cm}|p{5.5cm}|p{6cm}|}
                    \hline
                    \rowcolor{lighter-grayer}
                    \textbf{Identificativo} & \textbf{Schema \gloxy{JSON}} & \textbf{Descrizione} \\
                    \hline
                    \endfirsthead
                    200 & 
                    \texttt{
                        \{ \newline 
                            "id": int \newline 
                            "name": string \newline 
                            "issues": int \newline 
                            "violations": int \newline 
                            "avg-pending": float \newline 
                            "avg-resolved": float \newline 
                        \}
                    } 
                    & Risposta contenente:
                    \begin{itemize}
                        \item nome del customer
                        \item numero di issue aperte
                        \item numero di violazioni subite
                        \item tempo medio di presa in carico
                        \item tempo medio di risoluzione
                    \end{itemize}
                    \\ \hline
                    400 & \texttt{\{ \newline "message": string \newline \}} & I timestamp forniti non sono validi
                    \\ \hline
                    404 & \texttt{\{ \newline "message": string \newline \}} & L'id del customer fornito non è stato trovato
                    \\ \hline
                \end{longtable}
            \end{center}
            \item \textbf{HTTP headers}: 
            \begin{itemize}
                \item \textbf{Content-Type:} \texttt{application/json} ;
            \end{itemize}
        \end{itemize}
    
    
    \newpage
    \subsection{/issue/\{issue-id\}}
        \begin{itemize}
            \item \textbf{Tipo del metodo}: GET;
            \item \textbf{Descrizione}: endpoint per l'ottenimento delle informazioni rispetto a un issue.
            \item \textbf{Parametri}: \\
            Nessuno
            \item \textbf{Risposte}: 
            \begin{center}
                \rowcolors{2}{lightest-grayest}{white}
                \begin{longtable}{|p{2.5cm}|p{5.5cm}|p{6cm}|}
                    \hline
                    \rowcolor{lighter-grayer}
                    \textbf{Identificativo} & \textbf{Schema \gloxy{JSON}} & \textbf{Descrizione} \\
                    \hline
                    \endfirsthead
                    200 & 
                    \texttt{
                        \{ \newline 
                        "id": int \newline 
                        "customer": int \newline 
                        "status": string \newline 
                        "priority": string \newline 
                        "subject": string \newline 
                        "description": string \newline 
                        "start-date": string(yyyy-mm-gg) \newline 
                        "time-company": int \newline 
                        "time-client": int \newline 
                        \}
                    } 
                    & Risposta contenente :
                    \begin{itemize}
                        \item id dell'issue
                        \item nome del customer
                        \item stato dell'issue
                        \item priorità dell'issue
                        \item oggetto dell'issue
                        \item descrizione dell'issue
                        \item data di apertura dell'issue
                        \item tempo che è stato in carico a Euronovate
                        \item tempo che è stato in carico al Customer
                    \end{itemize}
                    \\ \hline
                    404 & \texttt{\{ \newline "message": string \newline \}} & L'id dell'issue fornito non è stato trovato
                    \\ \hline
                \end{longtable}
            \end{center}
            \item \textbf{HTTP headers}: 
            \begin{itemize}
                \item \textbf{Content-Type:} \texttt{application/json} ;
            \end{itemize}
        \end{itemize}
    
    
    
    
    \newpage
    \subsection{/issues}
        \begin{itemize}
            \item \textbf{Tipo del metodo}: GET;
            \item \textbf{Descrizione}: endpoint per ottenere informazioni statistiche sull'andamento nel tempo delle issue.
            \item \textbf{Parametri}: \\
                \begin{center}
                    \rowcolors{2}{lightest-grayest}{white}
                    \begin{longtable}{|p{4cm}|p{4cm}|p{6cm}|}
                        \hline
                        \rowcolor{lighter-grayer}
                        \textbf{Nome} & \textbf{Tipo} & \textbf{Descrizione} \\ \hline
                        \texttt{min} & Timestamp GMT+2 & Data minima dell'apertura di ticket da considerare \\ \hline
                        \texttt{max} & Timestamp GMT+2 & Data massima dell'apertura di ticket da considerare \\ \hline
                        \texttt{group-by} & string & String rappresentante la politica di raggruppamento; deve essere una tra:
                        \begin{itemize}
                            \item day
                            \item month (default)
                            \item year
                        \end{itemize}
                            \\ \hline
                    \end{longtable}
                \end{center}
            \item \textbf{Risposte}: 
            \begin{center}
                \rowcolors{2}{lightest-grayest}{white}
                \begin{longtable}{|p{2.5cm}|p{5.5cm}|p{6cm}|}
                    \hline
                    \rowcolor{lighter-grayer}
                    \textbf{Identificativo} & \textbf{Schema \gloxy{JSON}} & \textbf{Descrizione} \\
                    \hline
                    \endfirsthead
                    200 & 
                    \texttt{
                        [
                        \{ \newline 
                        "begin-date": string (yyyy-mm-gg) \newline 
                        "created-issues": int \newline 
                        "open-issues": int \newline 
                        "violations": int \newline 
                        \},\newline 
                        ...\newline 
                        ]
                    } 
                    & Risposta contenente una collezione di oggetti contenenti:
                    \begin{itemize}
                        \item la data di inizio periodo
                        \item numero di issue aperte in quel periodo
                        \item numero di issue aperte all'inizio di quel periodo 
                        \item numero di violazioni notificate in quel periodo
                    \end{itemize}
                    \\ \hline
                    400 & \texttt{\{ \newline "message": string \newline \}} & I timestamp forniti non sono validi o il campo di raggruppamento non contiene uno dei valori definiti.
                    \\ \hline
                \end{longtable}
            \end{center}
            \item \textbf{HTTP headers}: 
            \begin{itemize}
                \item \textbf{Content-Type:} \texttt{application/json} ;
            \end{itemize}
        \end{itemize}
    \newpage
    
    
    
    
    
    
    
    
    
    
\section{Sviluppo}
	In questa sezione verranno trattate la parti principali e più interessanti della codifica e realizzazione del progetto.
	\subsection{Tecnologie e strumenti}
		Durante la fase di sviluppo son stati necessari strumenti per la verifica e lo sviluppo dei vari componenti, quali:
		\begin{itemize}
			\item Java : linguaggio con il quale si è sviluppato il progetto per la maggior parte, principalmente la sua versione 11 LTS
			\item JSON : markup usato per la pubblicazione dei dati dell'API
			\item XML : markup usato per il salvataggio dei dati sui customers, sui canali di notifica e sugli orari di lavoro su disco
			\item Git : sistema di versionamento usato per il tracciamento del progetto
			\item CodeCommit : servizio di hosting online per repository che supporta Git, di proprietà AWS
			\item Spring Boot : framework principale del progetto a base Java, estremamente diffuso in ambito aziendale/lavorativo
			\item Redmine SDK : libreria usata per interfacciarsi con le API di Redmine
			\item Telegram SDK : libreria usata per interfacciarsi con le API di Telegram
			\item Spring Mail : libreria usata per interfacciarsi con server mail
			\item IntelliJ IDEA : IDE usato per lo sviluppo del progetto
			\item JUnit : libreria Java per lo sviluppo di test di unità
			\item Postman : applicazione per interrogazione e testing di API
			\item Docker : prodotto usato per il deploy finale del prodotto
			\item SQL : linguaggio per database basati sul modello relazionale, usato per il mantenimento dei dati, in particolare usato con DMBS  PostgreSQL
			\item pgAdmin : client per la gestione di DBMS PostgreSQL
		\end{itemize}
	\subsection{Ordine di sviluppo}
		Come descritto in precedenza, il prodotto prevede dei sistemi isolati tra di loro, ognuno con il proprio obiettivo. \\
		Appunto per la loro peculiarità di essere isolati, si è proceduto allo sviluppo di ognuno di essi in modo sequenziale, passando al successivo solo quando quello corrente è completo e funzionante. \\
		In particolare, lo sviluppo ha seguito il seguente ordine:
		\begin{enumerate}
			\item Base di dati
			\item Layer di persistenza
			\item Sistema di ottenimento
			\item Sistema di aggiornamento
			\item Sistema di analisi
			\item Sistema di notifica
			\item Sistema di pubblicazione
		\end{enumerate}
		Successivo allo sviluppo di ognuno di questi punti, si è svolto un allineamento con il tutor aziendale, per assicurarsi che il sistema prodotto fosse completo e funzionale.\\
		Seguendo questo ordine è stato più facile sviluppare i sistemi il più isolati possibili, e quindi concentrandosi sugli obiettivi di essi e il come potessero essere usati, ma non come sarebbero stati usati. \\
	\subsection{Differenze dalla progettazione}
		Per quanto il prodotto sviluppato assomigli molto a quello pensato durante la fase di progettazione, son state necessarie alcune modifiche, per correggere alcuni aspetti non previsti in fasi precedenti, in particolare:
		\begin{itemize}
			\item Sistema di aggiornamento: questo sistema non era previsto dalla progettazione, ma considerata la tarda notifica della granularità dei vincoli presenti nelle SLA, è stato necessario introdurlo per mantenere consistenti le non minimalità presenti nella base di dati.
			\item Algoritmo calcolo tempo su orario lavorativo: è stato necessario lo sviluppo di alcune classi di supporto per i vari sistemi per la gestione delle tempistiche con ottica lavorativa, quindi considerando un orario lavorativo.
			\item Algoritmo di invio di notifiche estendibile per le varie necessità dei canali, per evitare di andare a violare definiti dai rate limiter dei vari sistemi esterni.
			\item sviluppo di un orchestratore per i vari manager dei vari sistemi (Engine)
		\end{itemize}
		Questi nuovi componenti non hanno però portato un ritardo allo sviluppo in quanto fondamentali ma relativamente semplici da implementare. \\
		Considerate queste modifiche, il progetto finale ha un architettura che può essere descritta dal seguente schema:
		\begin{center}
			\includegraphics[keepaspectratio = true, width=16cm]{immagini/architettura-finale.png}
			\captionof{figure}{Architettura finale del progetto}
		\end{center}
		
	\subsection{Particolarità }
		Il prodotto è stato sviluppato con alcune caratteristiche ben definite fin dall'inizio come pilastri principali, quali:
		\begin{itemize}
			\item indipendenza dal provider esterno per l'ottenimento dei dati
			\item facile estensione dei canali di notifica disponibili
			\item configurabile tramite file XML
		\end{itemize}
		\subsubsection{Sistema di ottenimento}
			Il sistema di ottenimento doveva essere sviluppato tenendo a mente che un giorno si potesse migrare a un provider esterno diverso da Redmine, e che tale migrazione dovesse richiedere lo sviluppo del minor codice possibile. \\
			Tale obiettivo si è raggiunto con un esteso utilizzo di 2 design pattern, quali:
			\begin{itemize}
				\item Object Adapter
				\item Strategy
			\end{itemize}
			Tali design pattern si possono vedere applicati nel seguente UML che descrive l'effettiva implementazione di questo sistema:
   			\begin{center}
				\includegraphics[keepaspectratio = true, width=16cm]{immagini/ottenimento.png}
				\captionof{figure}{UML sistema di ottenimento dati}
			\end{center}
			In arancione si denota l'applicazione del design pattern Adapter, mentre in verde l'applicazione del design pattern Strategy. \\
			Come si può notare, la classe responsabile della logica di ottenimento dei dati, \texttt{AcquisitionManager}, si interfacci al sistema esterno di ottenimento dati tramite un interfaccia, la quale definisce i metodi necessari per la strategy da implementare, e che tale strategy gestisca dati (parametri e valore di ritorno) che a loro volta sono interfacce usate per l'implementazione del design pattern Adapter: son state poi sviluppate quindi le classi adapter necessarie per adattare i vari oggetti restituiti dall'SDK di Redmine, a tali interfacce.\\
			In questo modo, se un domani si decidesse di cambiare provider, sarà necessario l'implementazione delle strategy \texttt{ExternalAcquisitionService} e dei relativi adapter per i dati ottenuti, mentre la logica di gestione di tali dati rimarrà invariata, al mantenimento degli invarianti richiesti dalle interfacce.\\
		\subsubsection{Sistema di notifica}
			Il sistema di notifica doveva essere sviluppato in modo da permettere l'implementazione di nuovi sistemi di notifica.\\
			A tale requisito, si è scoperto in fase di codifica anche la necessità di permettere politiche di invio diverse per ogni canale, in quanto per esempio, Telegram non permette l'invio di più di 20 messaggi al minuto, mentre Gmail permette l'invio di al massimo 60 mail al minuto. \\
			Per far fronte al primo requisito, si è utilizzato il design pattern Strategy per lo sviluppo dei vari canali, mentre per il secondo si è usato il meccanismo del Double Dispatch fornito dal design pattern Visitor.\\
			Tali design pattern possono essere osservati nel seguente diagramma UML, che descrive l'effettiva implementazione di questo sistema:
			\begin{center}
				\includegraphics[keepaspectratio = true, width=16cm]{immagini/notifica.png}
				\captionof{figure}{UML sistema di notifica}
			\end{center}
			In giallo si può vedere l'applicazione del design pattern Visitor, che permetterà al sistema di ottenere un \texttt{ExecutorService} dedicato per ogni canale di notifica, e uno di default nel caso fossero presenti canali di notifica senza particolari esigenze di rate limiting. \\
			In particolare, per Telegram ritornerà un \texttt{ExecutorService} con pause di 3 secondi tra ogni \texttt{Runnable} fornitogli,  per Mail ritornerà un \texttt{ExecutorService} con pause di 0.5 secondi tra ogni \texttt{Runnable} fornitogli, mentre per il canale di default ritornerà un \texttt{CachedExecutorService} che permetterà il riutilizzo di \texttt{Thread} già istanziati.\\
			In verde invece si può vedere la strategy applicata ai due canali identificati durante la fase di analisi dei requisiti, la quale permetterà l'estensione a qualsiasi altro canale di notifica, a patto che implementi la sua interfaccia.\\
		\subsubsection{Gestione della configurazione}
			Il progetto richiedeva la possibilità di definire alcuni dati tramite file esterni. \\
			Alcune di queste informazioni sono:
			\begin{itemize}
				\item SLA dei customer (e SLA di default)
				\item orari di lavoro dei customer (e orario di default)
				\item canali di notifica per ogni customer (e canali di default)
				\item orario di lavoro dell'azienda
			\end{itemize}
			Tali file dovevano essere esterni al progetto così che, una volta creato il JAR e fatto il deploy, fosse possibile modificarli.\\
			Per ottenere ciò, son state create delle Factory per tali oggetti, che quando richiesto, andranno a deserializzare i file XML richiesti, con alcune politiche di fallback in caso di errore di lettura. \\
			All'avvio poi, basterà passare come parametro di esecuzione il path al quale sono presenti questi file.\\
		\subsubsection{Engine}
			Con \texttt{Engine}, come si può vedere nell'immagine di inizio capitolo, è l'orchestratore dei vari sistemi. \\
			Esso è responsabile dell'esecuzione dei vari manager dei sistemi nel ordine corretto, e nel caso, della gestione degli errori. \\
			Essendo i vari sistemi stati sviluppati per essere usati ad eventi, l'implementazione di questo servizio non dovrà altro che collegare i vari listeners dei manager, con i relativi eventi. \\
\iffalse			L'\texttt{Engine} sviluppato, infine, ha una logica di aggiornamento che potrebbe essere riassunta con il seguente schema:
			\begin{center}
				\includegraphics[keepaspectratio = true, width=16cm]{immagini/engine.png}
				\captionof{figure}{Logica di aggiornamento dell'Engine}
			\end{center} \fi
	\subsection{Extra}
		Considerato che lo sviluppo del progetto ha richiesto meno del previsto, si è proceduti allo sviluppo di feature aggiuntive esterne al progetto, quali:
		\begin{itemize}
			\item Dashboard Grafana per l'analisi dei dati presenti nella base di dati
			\item Docker container per il deploy dell'intero progetto, comprendente di engine, dashboard Grafana e database PosgreSQL
		\end{itemize}
		\subsubsection{Grafana}
			Grafana è uno progetto open-source che mira a semplificare la creazione di dashboard dinamiche per vari tipi di basi di dati. Supporta infatti basi di dati relazionali, a grafo e non relazionali.\\
			Si è proceduto quindi alla creazione di un DataSource per Grafana per la connessione al database usato, e successivamente a una Dashboard connessa a tale DataSource per la visualizzazione di alcuni highlight della base di dati. \\
			Tale dashboard è la seguente:
			\newpage
			\begin{center}
				\includegraphics[keepaspectratio = true, height=20cm]{immagini/dashboard.png}
				\captionof{figure}{Dashboard Grafana}
			\end{center}
			Essa mira a mostrare un'analisi istantanea della condizione della base di dati, come il numero di ticket ancora aperti o quanti son state chiusi nelle scorse 48 ore, e un analisi nel tempo dell'andamento dei ticket.
		\subsection{Docker}
			Essendo il progetto sviluppato, completo e funzionante, si è deciso di procedere al deploy. Sfortunatamente l'azienda non disponeva di server liberi nel quale procedere all'installazione dei vari servizi necessari per il funzionamento del prodotto, quindi si è proceduto alla creazione di Dockerfile per un deploy su un DockerEngine.\\
			In particolare, si è creato un Dockerfile per l'engine, con al suo interno tutti gli step necessari per il suo deploy, e un \texttt{docker-compose.yml} per il deploy di tutte e tre le parti; in questo file erano quindi presenti 3 service quali:
			\begin{itemize}
					\item \texttt{engine }: service per il deploy del prodotto sviluppato
					\item \texttt{database }: service per il deploy della base di dati, creazione del database e popolamento di esso
					\item \texttt{granafa }:  service per il deploy di Grafana e per l'importazione della sua DataSource e Dashboard
			\end{itemize}
			Grazie a ciò, si è permesso un deploy semplice e riproducibile in qualsiasi server, tramite il comando \texttt{docker compose up}, senza la necessità di interventi sul server effettivo. Ciò ha inoltre standardizzato le cartelle all'interno del container dell'engine, permettendo un più facile configurazione dei file XML esterni.\\
			Infine, questo ha permesso una più facile definizione delle variabili di ambiente necessarie per il corretto funzionamento dei vari sistemi.
			
	\subsection{Problematiche riscontrate}
		La fase di codifica si è conclusa con l'adempimento di tutti i requisiti individuati durante la fase di analisi dei requisiti, e i requisiti successivamente individuati durante la fase stessa di codifica, come Grafana, Docker e aggiunte descritte nel paragrafo precedente.\\
		Di seguito vengono esposti le problematiche principali riscontrate durante lo sviluppo di questo progetto, e la relativa soluzione adottata:
		 \begin{center}
			\rowcolors{2}{lightest-grayest}{white}
			\begin{longtable}{|p{7cm}|p{7cm}|}
				\hline
				\rowcolor{lighter-grayer}
				\textbf{Problema} & \textbf{Soluzione} \\
				\hline
				\endfirsthead
				Bug SDK Redmine che non permetteva l'inserimento di più filtri con la medesima chiave nella stessa richiesta & Risolto: inizialmente si aveva provato a risolvere tramite la Reflection di Java, andando a cambiare il comportamento del metodo che causava ciò, ma infine si è riusciti a ottenere il comportamento ottenuto anche con la libreria standard, usando un solo parametro\\ \hline
				File XML di configurazione esterni potevano essere invalidi o assenti & Risolto: si son previsti dei file contenenti le informazioni da usare "di default" in caso quelli principali fossero malformati o assenti, e se anch'essi avessero problemi, degli oggetti di default hardcoded nel programma, così da avere sempre uno stato funzionante dell'engine\\ \hline
				Lo storico delle issue non doveva sollevare notifiche & Risolto: si è previsto una logica di analisi che permette di evitare di inviare notifiche di segnalazione o violazione in caso quel ticket fosse stato già segnalato \\ \hline
				L'applicativo ha dipendenze sul sistema non triviali per il deploy & Risolto: inizialmente si aveva provato a creare uno script di installazione, che procedesse a installare tutti i sistemi necessari per l'esecuzione, ma infine si è optato per l'uso di Docker in quanto prodotto standard usato esattamente per questo obbiettivo \\ \hline
			\end{longtable}
		\end{center}             % Product Prototype
% !TEX encoding = UTF-8
% !TEX TS-program = pdflatex
% !TEX root = ../tesi.tex

%**************************************************************
\chapter{Verifica e validazione}
\label{cap:verifica-validazione}
%**************************************************************
\intro{In questo capitolo si descrivono i processi di verifica e validazione del prodotto, descrivendo i tool utilizzati per valutare il corretto funzionamento e la qualità del prodotto.} 
	\section{Verifica}
		Vengono di seguito descritti gli strumenti utilizzati per verificare la correttezza di tutto il materiale prodotto durante lo stage.
		\subsection{Documentazione}
			Durante lo stage c'è stata la necessità della stesura di alcuni documenti quali:
			\begin{itemize}
				\item Analisi dei Requisiti: documento mirato a riassumere i requisiti dello stage e individuare i casi d'uso del progetto;
				\item Progettazione Tecnica: documento mirato a definire una struttura generica dalla quale partire per lo sviluppo nella fase di codifica;
				\item Manuale Installazione: manuale contenente una guida testuale all'installazione del prodotto e al suo uso (come per esempio la gestione dei file XML);
				\item Manuale Manutenzione: manuale contenente una descrizione generica di come è implementato/strutturato il progetto, nel quale si vanno a spiegare potenziali punti di estensione e, se ci sono, eventuali problemi  presenti nel codice.
			\end{itemize}
			Per la redazione di tali documenti è stato usato \LaTeX, i cui file sorgenti son stati verificati tramite Spell Checker, estensione di VsCode per la verifica di errori ortografici all'interno di file sorgenti, permettendo come dizionari sia quello italiano che quello inglese.\\
			
		\subsection{Codice}
			Per quanto non fosse un requisito dello stage, durante la codifica, alcune parti del codice hanno richiesto la stesura di test di unità. \\
			Questo perché, per quanto il codice sia stato scritto seguendo il maggior numero di best practice possibile, ci son state parti particolarmente algoritmiche intrinsecamente poco leggibili. Per questo motivo son stati sviluppati dei test di unità automatici che vanno a verificare i requisiti che i metodi di queste classi devono rispettare, così da semplificare la manutenzione e modifica futura di tale codice. \\
			Per la redazione di tali test è stato usato \texttt{JUnit4}, libreria Java per appunto lo sviluppo di suite di test di unità.
			\begin{center}
				\includegraphics[keepaspectratio = true, width=16cm]{immagini/test.png}
				\captionof{figure}{Report suite test di unità}
			\end{center}
	\section{Validazione}
		Viene di seguito descritto il processo di validazione del materiale prodotto durante il periodo di stage.
		\subsection{Documentazione}
			La validazione  della documentazione prodotta durante lo stage è stata effettuata dal tutor aziendale Matteo Gnoato, assumendo il ruolo di committente: egli stabiliva se la documentazione fosse conforme e corretta rispetto a quanto atteso e, dopo aver chiesto delle eventuali correzioni, il documento veniva approvato e inviato a tutte le persone coinvolte ed interessate in questo progetto di stage.
		\subsection{Codice}
			La validazione del codice è avvenuta in due fasi.\\
			La prima, ufficiosa, è avvenuta con il tutor aziendale in meeting privati, nel quale si andava a ispezionare il codice per verificarne tutte le proprietà richieste, come leggibilità, testabilità, manutenibilità e correttezza. \\
			La seconda, ufficiale, l'ultimo giorno del progetto di stage, il giorno 10 Giugno 2021, con una demo, il cui scopo è stato quello di mostrare all'azienda ciò che si ha realizzato durante il periodo di stage, al fine di determinare il grado di soddisfazione dei requisiti stabiliti. Tale demo è avvenuta da remoto, con presenti il tutor aziendale e alcuni dipendenti di Euronovate che hanno partecipato all'individuazione dei requisiti (Quality Assurance) e al deploy dell'applicativo finale. \\
			             % Product Design Freeze e SOP
% !TEX encoding = UTF-8
% !TEX TS-program = pdflatex
% !TEX root = ../tesi.tex

%**************************************************************
\chapter{Documentazione}
\label{cap:documentazione}
%**************************************************************
\intro{In questo capitolo si descrive la documentazione prodotta per l'applicativo sviluppato durante il periodo di stage.} 
	\section{Codice}
		Il codice prodotto durante il periodo di stage è stato scritto interamente per lo sviluppo dell'engine, che consiste in un unico progetto Maven. \\
		Per la sua documentazione, si è deciso di seguire lo standard JavaDoc, in questo modo si può generare un sito con la documentazione di tutte le classi/metodi prodotti tramite \texttt{mvn site}. \\
		Tutte le classi e i metodi son stati quindi preceduti da un commento descrivente l'invariante, le precondizioni, gli input e infine l'output aspettato da essi, con eccezione del controller principale dell'API, che è stato documentato tramite un file Swagger descritto nel capitolo successivo. 
	\section{Manuali}
		Lo stage richiedeva che al termine della codifica, fossero redatti i seguenti due manuali:
		\begin{itemize}
			\item Manuale Manutentore
			\item Manuale Installazione
		\end{itemize}
		Nel primo documento si ha affrontato la manutenzione del progetto: in esso infatti si è andato a descrivere come è stato creato il progetto, e come si intende esser sviluppato. Dentro si può trovare una descrizione dei package, una descrizione ad alto livello della logica dell'engine e infine i punti di estensione, quindi quelle parti del codice che probabilmente in un futuro si vorrà sapere come estendere per aggiungere funzionalità (nel caso di questo progetto, come aggiungere canali di notifica e come cambiare / aggiungere un provider diverso da Redmine)\\
		Nel secondo invece si vuole descrivere il processo di deploy dell'applicativo. In esso infatti si possono trovare i requisiti hardware minimi del server di deploy, le configurazioni e i servizi che tale server deve avere, e infine una guida su come far partire l'applicativo. In quanto il progetto richiedeva anche conoscenze esterne, come il saper creare un Bot Telegram, o come trovare un'ID di un gruppo Telegram, si è inserito inoltre una guida all'ottenimento di questi dati, oltre che all'utilizzo dei vari file XML necessari per il corretto funzionamento dell'engine.
	\section{API}
		Per quanto riguarda l'API, è stato prodotto un file Swagger descrivente input e output di ogni endpoint, così da poter generare PDF di documentazione esportabili, e avere una chiara visuale di come deve essere usata tale API.
		Il file redatto, se interpretato dal sito Swagger, genera la seguente documentazione:
		\begin{center}
			\includegraphics[keepaspectratio = true, height=20cm]{immagini/swagger.png}
			\captionof{figure}{File Swagger API}
		\end{center}             % Documentazione
% !TEX encoding = UTF-8
% !TEX TS-program = pdflatex
% !TEX root = ../tesi.tex

%**************************************************************
\chapter{Conclusioni}
\label{cap:conclusioni}
%**************************************************************
\intro{In questo capitolo conclusivo viene analizzato retrospettivamente il progetto di stage, focalizzandosi sul raggiungimento degli obiettivi, sulle conoscenze acquisite e/o rinforzate, e su una valutazione personale di questo percorso.}
%**************************************************************
\section{Consuntivo finale}
	Nel complesso, lo stage ha avuto una durata di esattamente 320 ore come preventivate da piano di lavoro, con conclusione il 10/06/2021 con una presentazione e demo a vari componenti e tutor aziendale.\\


%**************************************************************
\section{Raggiungimento degli obiettivi}
	Come descritto nei capitoli precedenti, il prodotto soddisfa tutti i requisiti, sia obbligatori che desiderabili, identificati durante la fase di analisi dei requisiti. \\
	Va oltre a tali requisiti con Grafana e Docker, in quanto non erano stati individuati come requisiti all'inizio, ma le tempistiche hanno permesso anche il loro completamento e rilascio.

%**************************************************************
\section{Conoscenze acquisite}
	Per la realizzazione di questo progetto son state fondamentali le nozioni apprese durante il corso di studi, in particolare il corso Ingegneria del Software, per la gestione del progetto (da un punto di vista teorico) e per la parte di codifica (dal punto di vista pratico), sopratutto la conoscenza dei design pattern per rendere modulare e estendibile il progetto, e il corso di Basi di Dati, per la creazione e progettazione della base di dati. \\
	Oltre alle conoscenze apprese grazie al corso di studi, son state fondamentali anche conoscenze extra, apprese durante lo stage, come:
	\begin{itemize}
		\item Spring: essendo un framework usato globalmente, nei modi più disparati, è estremamente grande e relativamente complesso rispetto a suoi concorrenti di altri linguaggi, ma è stato fondamentale per lo stage per lo sviluppo dell'applicativo
		\item Docker: applicativo utile per il deploy, anch'esso però relativamente complesso a priva vista, fondamentale per il rilascio dell'applicativo sviluppato
		\item Grafana: programma altamente customizzabile, facile da usare, con un sacco di feature pronte out-of-the-box, essenziale per la creazione della dashboard di analisi della base di dati
		\item PostgreSQL: per quanto anch'esso sia di base SQL, come MySQL, son state necessarie nozioni extra per il suo uso per la realizzazione della base di dati del progetto 
	\end{itemize}
	

%**************************************************************
\section{Valutazione personale}
	Il progetto di stage offerto da Euronovate, a mio parere, si è svolto in modo ottimo. Per quanto non elementare, è stato correttamente ponderato per le 320 ore lavorative previste, e l'azienda e il tutor, Matteo Gnoato, per quanto in un momento abbastanza impegnativo, son sempre stati disponibili per chiarimenti o incontri in caso di problemi. \\
	Personalmente ho trovato la parte iniziale, fino alla progettazione tecnica,  un po lenta, in quanto ancora improntata su un modello "in presenza", e che quindi richiedesse la configurazione della postazione di lavoro, e avendo un po di dimestichezza già con gli strumenti usati durante lo stage, anche la parte di esplorazione delle tecnologie è stata un po lenta, ma per il resto non c'è stato alcun problema. \\
	Per quanto mi sia sempre piaciuto programmare, durante lo stage ho capito che in se, programmare tutto il giorno, non è veramente la mia più grande passione, e questo sicuramente giocherà un ruolo importante nella scelta del corso magistrale che deciderò di intraprendere.
             % Conclusioni
% \appendix                               
% \input{capitoli/capitolo-A}             % Appendice A

\setcounter{secnumdepth}{0} % No section number
\setcounter{tocdepth}{0} % No section number


\chapter{Glossario}

\setcounter{secnumdepth}{1} % No section number
\setcounter{tocdepth}{3} % No section number
\subsection{A}
\subsubsection{API}
Interfaccia di programmazione delle applicazioni, che mira a spiegare come utilizzare il servizio dall'esterno (nel nostro caso, è un API web).
\subsubsection{API key}
Metodo di autenticazione ad un API esterna tramite una key univoca fornita dal servizio target.
\subsection{C}
\subsubsection{Customer}
Con Customers ci si riferisce ai clienti, in particolare ai "progetti" aperti sotto il macroprogetto "Customers" dentro Redmine al momento dell'analisi dei requisiti.
\subsection{I}
\subsubsection{Issue}
Guarda "Ticket".
\subsection{R}
\subsubsection{Redmine}
Software utilizzato internamente a Euronovate per la gestione di progetti e ticket.
\subsection{S}
\subsubsection{S.L.A.}
Guarda "Service Level Agreement".
\subsubsection{Service Level Agreement}
Sono strumenti contrattuali attraverso i quali si definiscono le metriche di servizio (nel nostro caso per esempio il tempo massimo di risoluzione di un ticket, o il tempo massimo di presa in carico).

\subsection{T}
\subsubsection{Telegram}
Servizio di messaggistica istantanea (preso in considerazione dal progetto come potenziale mezzo di notifica).
\subsubsection{Ticket}
Sinonimo di Issue, si intende una qualsiasi segnalazione aperta da parte di un Cusotmer.             % Appendice A

%**************************************************************
% Materiale finale
%**************************************************************
\backmatter
\printglossaries
% !TEX encoding = UTF-8
% !TEX TS-program = pdflatex
% !TEX root = ../tesi.tex

%**************************************************************
% Bibliografia
%**************************************************************

\cleardoublepage
\chapter{Bibliografia}
\section{Riferimenti alle tecnologie}
Di seguito vengono riportati i riferimenti ai vari tool e tecnologie previste per lo sviluppo del progetto:
\begin{enumerate}
	\item[ {[}1{]} ] Java: \url{https://docs.oracle.com/en/java/javase/15/docs/api/index.html}
	\item[ {[}2{]} ] API Redmine: \url{https://www.redmine.org/projects/redmine/wiki/rest_api}
	\item[ {[}3{]} ] SDK Redmine: \url{https://github.com/taskadapter/redmine-java-api}
	\item[ {[}4{]} ] Bot API Telegram: \url{https://core.telegram.org/bots}
	\item[ {[}5{]} ] SDK Bot Telegram: \url{https://github.com/rubenlagus/TelegramBots}
	\item[ {[}6{]} ] SDK Spring Mail: \url{https://mvnrepository.com/artifact/org.springframework.boot/spring-boot-starter-mail}
\end{enumerate}

\nocite{*}
% Stampa i riferimenti bibliografici
\printbibliography[heading=subbibliography,title={Riferimenti bibliografici},type=book]

% Stampa i siti web consultati
\printbibliography[heading=subbibliography,title={Siti web consultati},type=online]


\end{document}

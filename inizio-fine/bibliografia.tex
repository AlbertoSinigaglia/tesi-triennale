% !TEX encoding = UTF-8
% !TEX TS-program = pdflatex
% !TEX root = ../tesi.tex

%**************************************************************
% Bibliografia
%**************************************************************

\cleardoublepage
\chapter{Bibliografia}
\section{Riferimenti alle tecnologie}
Di seguito vengono riportati i riferimenti ai vari tool e tecnologie previste per lo sviluppo del progetto:
\begin{enumerate}
	\item[ {[}1{]} ] Java: \url{https://docs.oracle.com/en/java/javase/15/docs/api/index.html}
	\item[ {[}2{]} ] API Redmine: \url{https://www.redmine.org/projects/redmine/wiki/rest_api}
	\item[ {[}3{]} ] SDK Redmine: \url{https://github.com/taskadapter/redmine-java-api}
	\item[ {[}4{]} ] Bot API Telegram: \url{https://core.telegram.org/bots}
	\item[ {[}5{]} ] SDK Bot Telegram: \url{https://github.com/rubenlagus/TelegramBots}
	\item[ {[}6{]} ] SDK Spring Mail: \url{https://mvnrepository.com/artifact/org.springframework.boot/spring-boot-starter-mail}
\end{enumerate}

\nocite{*}
% Stampa i riferimenti bibliografici
\printbibliography[heading=subbibliography,title={Riferimenti bibliografici},type=book]

% Stampa i siti web consultati
\printbibliography[heading=subbibliography,title={Siti web consultati},type=online]

